%% Generated by Sphinx.
\def\sphinxdocclass{jupyterBook}
\documentclass[letterpaper,10pt,english]{jupyterBook}
\ifdefined\pdfpxdimen
   \let\sphinxpxdimen\pdfpxdimen\else\newdimen\sphinxpxdimen
\fi \sphinxpxdimen=.75bp\relax
\ifdefined\pdfimageresolution
    \pdfimageresolution= \numexpr \dimexpr1in\relax/\sphinxpxdimen\relax
\fi
%% let collapsible pdf bookmarks panel have high depth per default
\PassOptionsToPackage{bookmarksdepth=5}{hyperref}
%% turn off hyperref patch of \index as sphinx.xdy xindy module takes care of
%% suitable \hyperpage mark-up, working around hyperref-xindy incompatibility
\PassOptionsToPackage{hyperindex=false}{hyperref}
%% memoir class requires extra handling
\makeatletter\@ifclassloaded{memoir}
{\ifdefined\memhyperindexfalse\memhyperindexfalse\fi}{}\makeatother

\PassOptionsToPackage{warn}{textcomp}

\catcode`^^^^00a0\active\protected\def^^^^00a0{\leavevmode\nobreak\ }
\usepackage{cmap}
\usepackage{fontspec}
\defaultfontfeatures[\rmfamily,\sffamily,\ttfamily]{}
\usepackage{amsmath,amssymb,amstext}
\usepackage{polyglossia}
\setmainlanguage{english}



\setmainfont{FreeSerif}[
  Extension      = .otf,
  UprightFont    = *,
  ItalicFont     = *Italic,
  BoldFont       = *Bold,
  BoldItalicFont = *BoldItalic
]
\setsansfont{FreeSans}[
  Extension      = .otf,
  UprightFont    = *,
  ItalicFont     = *Oblique,
  BoldFont       = *Bold,
  BoldItalicFont = *BoldOblique,
]
\setmonofont{FreeMono}[
  Extension      = .otf,
  UprightFont    = *,
  ItalicFont     = *Oblique,
  BoldFont       = *Bold,
  BoldItalicFont = *BoldOblique,
]



\usepackage[Bjarne]{fncychap}
\usepackage[,numfigreset=1,mathnumfig]{sphinx}

\fvset{fontsize=\small}
\usepackage{geometry}


% Include hyperref last.
\usepackage{hyperref}
% Fix anchor placement for figures with captions.
\usepackage{hypcap}% it must be loaded after hyperref.
% Set up styles of URL: it should be placed after hyperref.
\urlstyle{same}


\usepackage{sphinxmessages}



        % Start of preamble defined in sphinx-jupyterbook-latex %
         \usepackage[Latin,Greek]{ucharclasses}
        \usepackage{unicode-math}
        % fixing title of the toc
        \addto\captionsenglish{\renewcommand{\contentsname}{Contents}}
        \hypersetup{
            pdfencoding=auto,
            psdextra
        }
        % End of preamble defined in sphinx-jupyterbook-latex %
        

\title{ECON7001}
\date{Dec 21, 2023}
\release{}
\author{Fedor Iskhakov}
\newcommand{\sphinxlogo}{\vbox{}}
\renewcommand{\releasename}{}
\makeindex
\begin{document}

\pagestyle{empty}
\sphinxmaketitle
\pagestyle{plain}
\sphinxtableofcontents
\pagestyle{normal}
\phantomsection\label{\detokenize{index::doc}}


\begin{DUlineblock}{0em}
\item[] \sphinxstylestrong{\Large Preliminary schedule}
\end{DUlineblock}


\begin{savenotes}\sphinxattablestart
\centering
\begin{tabulary}{\linewidth}[t]{|T|T|T|T|}
\hline
\sphinxstyletheadfamily 
\sphinxAtStartPar
Week
&\sphinxstyletheadfamily 
\sphinxAtStartPar
Date
&\sphinxstyletheadfamily 
\sphinxAtStartPar
Topic
&\sphinxstyletheadfamily 
\sphinxAtStartPar
Notes
\\
\hline
\sphinxAtStartPar
1
&
\sphinxAtStartPar
Feb 19
&
\sphinxAtStartPar
{\hyperref[\detokenize{00.intro::doc}]{\sphinxcrossref{\DUrole{doc,std,std-doc}{Introduction}}}}
&
\sphinxAtStartPar

\\
\hline
\sphinxAtStartPar
2
&
\sphinxAtStartPar
Feb 26
&
\sphinxAtStartPar
\DUrole{xref,myst}{}
&
\sphinxAtStartPar
Tutorials start
\\
\hline
\sphinxAtStartPar
3
&
\sphinxAtStartPar
Mar 4
&
\sphinxAtStartPar
\DUrole{xref,myst}{}
&
\sphinxAtStartPar

\\
\hline
\sphinxAtStartPar
4
&
\sphinxAtStartPar
Mar 11
&
\sphinxAtStartPar
\DUrole{xref,myst}{}
&
\sphinxAtStartPar

\\
\hline
\sphinxAtStartPar
5
&
\sphinxAtStartPar
Mar 18
&
\sphinxAtStartPar
\DUrole{xref,myst}{}
&
\sphinxAtStartPar

\\
\hline
\sphinxAtStartPar
6
&
\sphinxAtStartPar
Mar 25
&
\sphinxAtStartPar
\DUrole{xref,myst}{}
&
\sphinxAtStartPar

\\
\hline
\sphinxAtStartPar
Break
&
\sphinxAtStartPar

&
\sphinxAtStartPar

&
\sphinxAtStartPar
2 weeks
\\
\hline
\sphinxAtStartPar
7
&
\sphinxAtStartPar
Apr 15
&
\sphinxAtStartPar
\DUrole{xref,myst}{}
&
\sphinxAtStartPar

\\
\hline
\sphinxAtStartPar
8
&
\sphinxAtStartPar
Apr 22
&
\sphinxAtStartPar
\DUrole{xref,myst}{}
&
\sphinxAtStartPar

\\
\hline
\sphinxAtStartPar
9
&
\sphinxAtStartPar
Apr 29
&
\sphinxAtStartPar
\DUrole{xref,myst}{}
&
\sphinxAtStartPar
TBA
\\
\hline
\sphinxAtStartPar
10
&
\sphinxAtStartPar
May 6
&
\sphinxAtStartPar
\DUrole{xref,myst}{}
&
\sphinxAtStartPar

\\
\hline
\sphinxAtStartPar
11
&
\sphinxAtStartPar
May 13
&
\sphinxAtStartPar
\DUrole{xref,myst}{}
&
\sphinxAtStartPar

\\
\hline
\sphinxAtStartPar
12
&
\sphinxAtStartPar
May 20
&
\sphinxAtStartPar
\DUrole{xref,myst}{}
&
\sphinxAtStartPar

\\
\hline
\end{tabulary}
\par
\sphinxattableend\end{savenotes}

\begin{DUlineblock}{0em}
\item[] \sphinxstylestrong{\large ANU course pages}
\end{DUlineblock}

\sphinxAtStartPar
\sphinxhref{https://wattlecourses.anu.edu.au/course/view.php?id=TBA}{Course Wattle page}
Schedule, announcements, teaching team contacts, recordings, assignement, grades

\sphinxAtStartPar
\sphinxhref{https://programsandcourses.anu.edu.au/2024/course/EMET7001}{Course overview} and
\sphinxhref{https://programsandcourses.anu.edu.au/2024/course/EMET7001/First\%20Semester/4098}{Class summary}
General course description in ANU Programs and Courses

\begin{DUlineblock}{0em}
\item[] \sphinxstylestrong{\large Assessment}
\end{DUlineblock}

\sphinxAtStartPar
TBA

\begin{DUlineblock}{0em}
\item[] \sphinxstylestrong{\large Tutoring team}
\end{DUlineblock}

\sphinxAtStartPar
TBA

\sphinxstepscope


\chapter{Welcome}
\label{\detokenize{00.intro:welcome}}\label{\detokenize{00.intro::doc}}
\sphinxAtStartPar
Course title: \sphinxstylestrong{“Mathematical Techniques for Economic Analysis”}
\begin{itemize}
\item {} 
\sphinxAtStartPar
Compulsory first math course in the \sphinxstyleemphasis{Master of Economics} program

\end{itemize}


\section{Plan for this lecture}
\label{\detokenize{00.intro:plan-for-this-lecture}}\begin{enumerate}
\sphinxsetlistlabels{\arabic}{enumi}{enumii}{}{.}%
\item {} 
\sphinxAtStartPar
Organization

\item {} 
\sphinxAtStartPar
Administrative topics

\item {} 
\sphinxAtStartPar
Course content

\item {} 
\sphinxAtStartPar
Self\sphinxhyphen{}learning materials

\end{enumerate}


\section{Instructor}
\label{\detokenize{00.intro:instructor}}
\sphinxAtStartPar
\sphinxstylestrong{Fedor Iskhakov}
Professor of Economics at RSE
\begin{itemize}
\item {} 
\sphinxAtStartPar
Office: 1021 HW Arndt Building

\item {} 
\sphinxAtStartPar
Email: \sphinxhref{mailto:fedor.iskhakov@anu.edu.au}{fedor.iskhakov@anu.edu.au}

\item {} 
\sphinxAtStartPar
Web: \sphinxhref{https://fedor.iskh.me}{fedor.iskh.me}

\item {} 
\sphinxAtStartPar
Contact hours: TBA

\end{itemize}


\section{Timetable}
\label{\detokenize{00.intro:timetable}}
\sphinxAtStartPar
\sphinxstylestrong{Face\sphinxhyphen{}to\sphinxhyphen{}face:}
\begin{itemize}
\item {} 
\sphinxAtStartPar
Lectures: TBA

\end{itemize}

\sphinxAtStartPar
\sphinxstylestrong{Online:}
\begin{itemize}
\item {} 
\sphinxAtStartPar
Echo\sphinxhyphen{}360 recordings on Wattle

\item {} 
\sphinxAtStartPar
All notes and materials on (domain TBA)

\end{itemize}

\sphinxAtStartPar
Face\sphinxhyphen{}to\sphinxhyphen{}face is strictly preferred


\section{Course web pages}
\label{\detokenize{00.intro:course-web-pages}}\begin{itemize}
\item {} 
\sphinxAtStartPar
\sphinxhref{https://wattlecourses.anu.edu.au/course/view.php?id=TBA}{Wattle}
Schedule, announcements, teaching team contacts, recordings, assignment, grades

\item {} 
\sphinxAtStartPar
{[}Online notes{]}(domain TBA)
Lecture notes, slides, assignment tasks

\item {} 
\sphinxAtStartPar
Lecture slides should appear online the previous day before the lecture

\item {} 
\sphinxAtStartPar
Details on assessment including the exam instructions will appear on Wattle

\end{itemize}


\section{Tutorials}
\label{\detokenize{00.intro:tutorials}}\begin{itemize}
\item {} 
\sphinxAtStartPar
Enrollments open on \sphinxstyleemphasis{Wattle}

\end{itemize}

\sphinxAtStartPar
Tutorial questions
\begin{itemize}
\item {} 
\sphinxAtStartPar
posted on the course website

\item {} 
\sphinxAtStartPar
not assessed, help you learn and prepare

\end{itemize}

\sphinxAtStartPar
Tutorials start on week 2


\section{Tutors}
\label{\detokenize{00.intro:tutors}}
\sphinxAtStartPar
TBA


\section{Prerequisites}
\label{\detokenize{00.intro:prerequisites}}
\sphinxAtStartPar
None specifically; \sphinxhref{https://programsandcourses.anu.edu.au/2024/course/EMET7001}{Course overview} and
\sphinxhref{https://programsandcourses.anu.edu.au/2024/course/EMET7001/First\%20Semester/4098}{Class summary}

\sphinxAtStartPar
Prior knowledge of maths, however, will be helpful:
\begin{itemize}
\item {} 
\sphinxAtStartPar
basic algebra

\item {} 
\sphinxAtStartPar
basic calculus

\item {} 
\sphinxAtStartPar
some idea of what a matrix is, etc.

\end{itemize}


\section{Focus?}
\label{\detokenize{00.intro:focus}}
\sphinxAtStartPar
This course will teach you the foundational mathematical concepts that you will need for other courses in your degree.  For example, \sphinxhref{https://programsandcourses.anu.edu.au/course/ECON6012}{Optimisation for Economics and Financial Economics (ECON6102)} is the compulsory second maths course in the \sphinxstyleemphasis{Master of Economics} program.  ECON6102 is a general course on mathematical modeling for economics and financial economics, but optimization will be an important and recurring theme.


\section{Assessment}
\label{\detokenize{00.intro:assessment}}
\sphinxAtStartPar
TBA


\section{Questions}
\label{\detokenize{00.intro:questions}}\begin{enumerate}
\sphinxsetlistlabels{\arabic}{enumi}{enumii}{}{.}%
\item {} 
\sphinxAtStartPar
Administrative questions: RSE admin

\end{enumerate}
\begin{itemize}
\item {} 
\sphinxAtStartPar
\sphinxstylestrong{Bronwyn Cammack} Senior School Administrator

\item {} 
\sphinxAtStartPar
Email: \sphinxhref{mailto:enquiries.rse@anu.edu.au}{enquiries.rse@anu.edu.au}

\item {} 
\sphinxAtStartPar
“I can not register for the tutorial group”

\end{itemize}
\begin{enumerate}
\sphinxsetlistlabels{\arabic}{enumi}{enumii}{}{.}%
\setcounter{enumi}{1}
\item {} 
\sphinxAtStartPar
Content related questions: please, refer to the tutors

\end{enumerate}
\begin{itemize}
\item {} 
\sphinxAtStartPar
“I don’t understand why this function is convex”

\end{itemize}
\begin{enumerate}
\sphinxsetlistlabels{\arabic}{enumi}{enumii}{}{.}%
\setcounter{enumi}{2}
\item {} 
\sphinxAtStartPar
Other questions: to Fedor

\end{enumerate}
\begin{itemize}
\item {} 
\sphinxAtStartPar
“I’m working hard but still can not keep up”

\item {} 
\sphinxAtStartPar
“Can I please have extra assignment for more practice”

\end{itemize}


\section{Attendance}
\label{\detokenize{00.intro:attendance}}\begin{itemize}
\item {} 
\sphinxAtStartPar
Please, \sphinxstylestrong{do not} use email for \sphinxstyleemphasis{instructional} questions\textbackslash{}Instead make use of the office hours

\item {} 
\sphinxAtStartPar
Attendance of tutorials is \sphinxstyleemphasis{very highly} recommended\\
You will make your life much easier this way

\item {} 
\sphinxAtStartPar
Attendance of lectures is \sphinxstyleemphasis{highly} recommended\\
But not mandatory

\end{itemize}


\section{Comments for lectures notes/slides}
\label{\detokenize{00.intro:comments-for-lectures-notes-slides}}\begin{itemize}
\item {} 
\sphinxAtStartPar
Cover exactly what you are required to know

\item {} 
\sphinxAtStartPar
Code inserts are the exception, they are not assessable

\end{itemize}

\sphinxAtStartPar
In particular, you need to know:
\begin{itemize}
\item {} 
\sphinxAtStartPar
The definitions from the notes

\item {} 
\sphinxAtStartPar
The facts from the notes

\item {} 
\sphinxAtStartPar
How to apply facts and definitions

\end{itemize}

\sphinxAtStartPar
If a concept in not in the lecture notes, it is not assessable


\section{Definitions and facts}
\label{\detokenize{00.intro:definitions-and-facts}}
\sphinxAtStartPar
The lectures notes/slides are full of definitions and facts.

\begin{sphinxadmonition}{note}{Definition}

\sphinxAtStartPar
Functions \(f: \mathbb{R} \rightarrow \mathbb{R}\) is called \sphinxstyleemphasis{continuous at} \(x\) if, for any sequence \(\{x_n\}\) converging to \(x\), we have \(f(x_n) \rightarrow f(x)\).
\end{sphinxadmonition}

\sphinxAtStartPar
Possible exam question: “Show  that if functions \(f\) and \(g\) are continuous at \(x\), so is \(f+g\).”

\sphinxAtStartPar
You should start the answer with the definition of continuity:

\sphinxAtStartPar
“Let \(\{x_n\}\) be any sequence converging to \(x\). We need to show that \(f(x_n) + g(x_n) \rightarrow f(x) + g(x)\). To see this, note that …”


\section{Facts}
\label{\detokenize{00.intro:facts}}
\sphinxAtStartPar
In the lecture notes/slides you will often see

\begin{sphinxadmonition}{note}{Fact}

\sphinxAtStartPar
The only \(N\)\sphinxhyphen{}dimensional subset of \(\mathbb{R}^N\) is \(\mathbb{R}^N\).
\end{sphinxadmonition}

\sphinxAtStartPar
This means either:
\begin{itemize}
\item {} 
\sphinxAtStartPar
theorem

\item {} 
\sphinxAtStartPar
proposition

\item {} 
\sphinxAtStartPar
lemma

\item {} 
\sphinxAtStartPar
true statement

\end{itemize}

\sphinxAtStartPar
All well known results. You need to remember them, have some intuition for, and be able to apply.


\section{Note on Assessments}
\label{\detokenize{00.intro:note-on-assessments}}
\sphinxAtStartPar
Assessable = definitions and facts + a few simple steps of logic

\sphinxAtStartPar
Exams and tests will award:
\begin{itemize}
\item {} 
\sphinxAtStartPar
Hard work

\item {} 
\sphinxAtStartPar
Deeper understanding of the concepts

\end{itemize}

\sphinxAtStartPar
In each question there will be an \sphinxstyleemphasis{easy} path to the solution

\sphinxstepscope


\chapter{An introduction to economics}
\label{\detokenize{01.intro_to_economics:an-introduction-to-economics}}\label{\detokenize{01.intro_to_economics::doc}}

\section{Sources and reading guide}
\label{\detokenize{01.intro_to_economics:sources-and-reading-guide}}\begin{itemize}
\item {} 
\sphinxAtStartPar
Alchian, AA and WR Allen (1972), \sphinxstyleemphasis{University economics: Elements of inquiry}, Wadsworth Publishing Company, USA: Chapters 1, 2, 3, 4, and 20 (pp. 2–52 and 384–404).

\item {} 
\sphinxAtStartPar
Alchian, AA, and WR Allen (1983), \sphinxstyleemphasis{Exchange and production: Competition, coordination and control (third edition)}, Wadsworth Publishing Company, USA: Chapter 1 (pp. 1–12).

\item {} 
\sphinxAtStartPar
Ausubel, LM, and RJ Deneckere (1993), “A generalized theorem of the maximum”, \sphinxstyleemphasis{Economic Theory 3(1)}, January, pp. 99–107.

\item {} 
\sphinxAtStartPar
Case, KE, RC Fair, and SM Oster (2017), \sphinxstyleemphasis{Principles of Economics (Twelfth Edition) (Global Edition)}, Pearson Education, Italy: Chapters 1 and 2 (pp. 35–75).

\item {} 
\sphinxAtStartPar
Frank, RH (2006), \sphinxstyleemphasis{Microeconomics and behavior (sixth edition)}, McGraw\sphinxhyphen{}Hill, USA: Chapter 1 (pp. 3–26).

\item {} 
\sphinxAtStartPar
Gans, J, S King, N Byford, and NG Mankiw (2018), \sphinxstyleemphasis{Principles of microeconomics (seventh Asia\sphinxhyphen{}Pacific edition)}, Cengage Learning Australia, China: Chapters 1, 2, and 3 (pp. 4–67).

\item {} 
\sphinxAtStartPar
Gravelle, H, and R Rees (2004), \sphinxstyleemphasis{Microeconomics (third edition)}, Pearson Education, United Kingdom: Chapter 1 (pp. 1–10).

\item {} 
\sphinxAtStartPar
Hamermesh, DS (2006), \sphinxstyleemphasis{Economics is everywhere (second edition)}, McGraw\sphinxhyphen{}Hill\sphinxhyphen{}Irwin, USA: Chapter 1 (pp. 3–14).

\item {} 
\sphinxAtStartPar
Heyne, P (2000), \sphinxstyleemphasis{A student’s guide to economics}, Edited by JA Eglarz, Intercollegiate Studies Institute (ISI) Books, USA.

\item {} 
\sphinxAtStartPar
Heyne, PL, PJ Boettke, and DL Prychitko (2014), \sphinxstyleemphasis{The economic way of thinking (thirteenth edition)}, The Pearson New International Edition, Pearson Education, USA: Chapters 1 and 2 (pp. 1–44).

\item {} 
\sphinxAtStartPar
Hirshleifer, J, A Glazer, and D Hirshleifer (2005), \sphinxstyleemphasis{Price theory and applications: Decisions, markets, and information (seventh edition)}, Cambridge University Press, USA: Chapter 1 (pp. 2–26).

\item {} 
\sphinxAtStartPar
Kunimoto, T (2010), \sphinxstyleemphasis{Lecture notes on mathematics for economists}, Unpublished, McGill University, Canada, 18 May, Page 6.

\item {} 
\sphinxAtStartPar
Malinvaude, E. (1972), \sphinxstyleemphasis{Lectures on microeconomic theory}, Advanced Textbooks in Economics Volume 2, North Holland Publishing Company, Scotland, Translated by Mrs. A. Silvey: Page 1.

\item {} 
\sphinxAtStartPar
Mankiw, NG (2003), \sphinxstyleemphasis{Macroeconomics (fifth edition)}, Worth Publishers, USA: Chapter 1 (pp. 2\sphinxhyphen{}14).

\item {} 
\sphinxAtStartPar
Perloff, JM (2014), \sphinxstyleemphasis{Microeconomics with calculus (third edition) (global edition)}, Pearson Education Limited, USA: Chapter 1 (pp. 23–30).

\item {} 
\sphinxAtStartPar
Robbins, LC (1984), \sphinxstyleemphasis{An essay on the nature and significance of economic science (third edition)}, With a foreword by WJ Baumol, New York University Press, Hong Kong. (The first edition of this book was published in 1932.)

\item {} 
\sphinxAtStartPar
Vohra, RV (2005), \sphinxstyleemphasis{Advanced mathematical economics}, Routledge, The United Kingdom: The Preface only.

\item {} 
\sphinxAtStartPar
Waud, RN, P Maxwell and J Bonnici (1989), \sphinxstyleemphasis{Macroeconomics (Australian edition)}, Harper and Row Publishers, Australia: Chapters 8\sphinxhyphen{}10 (pp. 169\sphinxhyphen{}249).

\end{itemize}


\section{What is economics?}
\label{\detokenize{01.intro_to_economics:what-is-economics}}
\sphinxAtStartPar
Economists try to explain social phenomena in terms of the behaviour of an individual who is confronted with scarcity and the interaction of that individual with other individuals who also face scarcity. This is perhaps best captured by Malinvaude’s definition of economics:
\begin{quote}

\sphinxAtStartPar
\sphinxstyleemphasis{“· · · economics is the science which studies how scarce resources are employed for the satisfaction of the needs of men living in society: on the one hand, it is interested in the essential operations of production, distribution and consumption of goods, and on the other hand, in the institutions and activities whose object
it is to facilitate these operations.” (Italics in original.)}
\end{quote}

\sphinxAtStartPar
– (From page one of Malinvaude, E. (1972), Lectures on microeconomic theory, Advanced Textbooks in Economics Volume 2, North Holland Publishing Company, Scotland, translated by Mrs. A. Silvey.)

\begin{sphinxadmonition}{note}{Note:}
\sphinxAtStartPar
A definition of economics along these lines (that is, one that emphasises the importance of scarcity) can be traced back at least as far as Lord Lionel Robbins’ justifiably famous “essay on the nature and significance of economic science”. Chapter one of this essay contains a very nice discussion of the definition of economics and its history.
\begin{itemize}
\item {} 
\sphinxAtStartPar
The first edition of this essay was published in 1932.

\item {} 
\sphinxAtStartPar
The third edition of this essay was published in 1984.

\end{itemize}
\end{sphinxadmonition}


\section{Core components of economics}
\label{\detokenize{01.intro_to_economics:core-components-of-economics}}

\subsection{The presence of scarcity}
\label{\detokenize{01.intro_to_economics:the-presence-of-scarcity}}\begin{itemize}
\item {} 
\sphinxAtStartPar
This is the defining feature of economics.

\item {} 
\sphinxAtStartPar
It is this feature that distinguishes economics from other social sciences.

\item {} 
\sphinxAtStartPar
In the absence of scarcity, economics would either not exist or look very different.

\end{itemize}


\subsection{The behaviour of an individual who is faced with scarcity}
\label{\detokenize{01.intro_to_economics:the-behaviour-of-an-individual-who-is-faced-with-scarcity}}\begin{itemize}
\item {} 
\sphinxAtStartPar
This involves the individual making a choice from a set of available (or feasible) alternatives.

\item {} 
\sphinxAtStartPar
The need to make a choice implies the existence of foregone alternatives and hence a cost.

\item {} 
\sphinxAtStartPar
The opportunity cost of something is the value of the best of the foregone alternatives.

\item {} 
\sphinxAtStartPar
Individual choice is often modelled using “constrained optimisation” techniques.

\end{itemize}


\subsection{The interaction of individuals that face scarcity}
\label{\detokenize{01.intro_to_economics:the-interaction-of-individuals-that-face-scarcity}}\begin{itemize}
\item {} 
\sphinxAtStartPar
Economic equilibrium (eg competitive equilibrium and Nash equilibrium).

\item {} 
\sphinxAtStartPar
When does a system of equations have at least one solution?

\item {} 
\sphinxAtStartPar
How do we find such a solution (if it exists)?

\item {} 
\sphinxAtStartPar
How does any such solution vary with changes in the parameters (exogenous variables) of the economic system being studied? (This is known as comparative statics.)

\item {} 
\sphinxAtStartPar
Use techniques from linear algebra and (for nonlinear cases) fixed point theorems.

\end{itemize}


\subsection{What is scarcity?}
\label{\detokenize{01.intro_to_economics:what-is-scarcity}}
\sphinxAtStartPar
Scarcity basically means that the availability of an item is limited
relative to the desired uses of that item.  Some important examples include:
\begin{itemize}
\item {} 
\sphinxAtStartPar
Scarcity of income or wealth (a budget constraint);

\item {} 
\sphinxAtStartPar
Scarcity of time (a time constraint);

\item {} 
\sphinxAtStartPar
Scarcity of productive resources and technological limitations (a production possibilities constraint).

\end{itemize}


\subsection{A budget constraint}
\label{\detokenize{01.intro_to_economics:a-budget-constraint}}
\sphinxAtStartPar
Suppose that there are two goods: Good one and good two. The price per unit of good one is \(p_1\) and the price per unit of good two is \(p_2\). The quantity of good one purchased by a consumer is \(q_1\) and the quantity of good two purchased by a consumer is \(q_2\). Note that the consumer’s total expenditure is \(p_1 q_1 + p_2 q_2\).

\sphinxAtStartPar
Suppose that the consumer’s income is \(y\). Ignoring the possibility of borrowing money from somewhere, the consumer cannot spend more than his or her income. This restriction is known as the budget constraint for the consumer. It can be
represented mathematically by the inequality \(p_1 q_1 + p_2 q_2 \leqslant y\). We typically also impose non\sphinxhyphen{}negativity constraints of the form \(q_1 \geqslant 0\) and \(q_2 \geqslant 0\).

\begin{figure}[htbp]
\centering
\capstart

\noindent\sphinxincludegraphics[width=0.800\linewidth]{{budget_line}.png}
\caption{The budget line}\label{\detokenize{01.intro_to_economics:id1}}\end{figure}

\begin{figure}[htbp]
\centering
\capstart

\noindent\sphinxincludegraphics[width=0.800\linewidth]{{budget_set}.png}
\caption{The budget set}\label{\detokenize{01.intro_to_economics:id2}}\end{figure}


\subsection{A time constraint}
\label{\detokenize{01.intro_to_economics:a-time-constraint}}
\sphinxAtStartPar
Suppose that there are activities that can be undertaken: Labour (N) and Leisure (L).  The amount of time spent working is \(t_N\) and the amount of time spent relaxing is \(t_L\).  Note that the total time that an individual spends on the various activities is \(t_N + t_L\).

\sphinxAtStartPar
Suppose that the total time available to the individual is \(T\). If we assume that every unit of time must be used for some activity, and that only one activity can be undertaken at any point in time, then the individual faces a time constraint of the form \(t_N + t_L = T\). We typically also impose non\sphinxhyphen{}negativity constraints of the form \(t_N \geqslant 0\) and \(t_L \geqslant 0\).

\begin{figure}[htbp]
\centering
\capstart

\noindent\sphinxincludegraphics[width=0.800\linewidth]{{time_constraint}.png}
\caption{A time constraint}\label{\detokenize{01.intro_to_economics:id3}}\end{figure}


\subsection{A production possibilities constraint}
\label{\detokenize{01.intro_to_economics:a-production-possibilities-constraint}}
\sphinxAtStartPar
Suppose that there are only two goods that can be produced: Food and housing. Given the current state of technology and the current stock of productive resources, for any given feasible amount of one of the goods that is produced, there is a limit on the maximum amount of the other good that can also be produced.

\sphinxAtStartPar
If we graph the maximum amount of housing that can be produced for each feasible choice of an amount of food , then the resulting curve is known as the production possibilities frontier (PPF). This frontier can be represented by an equation of the form \(F(Q_F, Q_H) = 0\), or alternatively by an equation of the form \(Q_H = G(Q_F)\).

\sphinxAtStartPar
The PPF, along with the area under it (bounded by the axes because of non\sphinxhyphen{}negativity constraints on production), is known as the Production Possibilities Set. This is the set of all possible feasible combinations of food and housing that could be produced under the current technology and resource constraints.

\begin{figure}[htbp]
\centering
\capstart

\noindent\sphinxincludegraphics[width=0.800\linewidth]{{ppf}.png}
\caption{A production possibilities constraint}\label{\detokenize{01.intro_to_economics:id4}}\end{figure}


\section{Scarcity and choice}
\label{\detokenize{01.intro_to_economics:scarcity-and-choice}}
\sphinxAtStartPar
The presence of scarcity means that individuals and societies must make choices. When you are required to make a choice between various alternatives, you are implicitly giving up the opportunities that you do not select in order to obtain the opportunity that you do select.

\sphinxAtStartPar
In the context of a budget constraint, if an individual purchases one more unit of good X , then he must give up \((P_X / P_Y)\) units of good \(Y\). In the context of a time constraint, if an individual takes an additional hour of leisure, then he must reduce the time that he works by one hour. This means that he will earn less income. As such, he will have to reduce his consumption by some amount. In the context of a production possibilities constraint, if a society that is currently on its PPF decides to produce one more unit of good \(Y\), then it will need to produce fewer units of good \(X\).


\subsection{Choice and cost}
\label{\detokenize{01.intro_to_economics:choice-and-cost}}
\sphinxAtStartPar
When you select your most preferred option from the available alternatives, you are effectively giving up all of the other options that were available to you. The fact that you must give up alternative outcomes when you choose a particular option means that there is a cost associated with your choice.


\subsection{Opportunity cost}
\label{\detokenize{01.intro_to_economics:opportunity-cost}}
\sphinxAtStartPar
As we have seen, scarcity requires choices and choices impose costs. In effect, the existence of costs is intimately related to the presence of scarcity. Costs arise because you must give up some option (or options) in order to obtain something that you want.

\begin{sphinxadmonition}{note}{Definition}

\sphinxAtStartPar
The value to you of the best (most preferred by you) of the alternative options that you give up is known as the \sphinxstylestrong{opportunity cost} of the option that you select.
\end{sphinxadmonition}

\begin{sphinxadmonition}{note}{Note:}
\sphinxAtStartPar
In economics, the word “cost” should always be taken to mean “opportunity cost”. Note that this might not be the same as the accounting cost or the monetary cost of something.
\end{sphinxadmonition}


\subsection{The importance of constrained optimisation}
\label{\detokenize{01.intro_to_economics:the-importance-of-constrained-optimisation}}
\sphinxAtStartPar
Given that scarcity is the defining feature of economics, it should not be surprising that constrained optimisation plays a central role in economic analysis. Indeed, constrained optimisation is very much a “bread and butter” skill for economists. It would difficult to make a living as an economist without some knowledge of constrained optimisation techniques.

\sphinxAtStartPar
According to Ausubel and Deneckere (1993, p. 99):
\begin{quote}

\sphinxAtStartPar
\sphinxstyleemphasis{“Almost every economic problem involves the study of an agent’s optimal choice as a function of certain parameters or state variables. For example, demand theory is concerned with an agent’s optimal consumption as a function of prices and income, while capital theory studies the optimal investment rule as a function of the existing capital stock.”}
\end{quote}


\subsection{Economic interactions between individuals}
\label{\detokenize{01.intro_to_economics:economic-interactions-between-individuals}}
\sphinxAtStartPar
Most people are not hermits. In general, individuals interact with other individuals on at least some occasions. These interactions are an important component of the subject matter of economics. Economics is, after all, one of the “social” sciences.

\sphinxAtStartPar
The interactions that occur between individuals that are important for economics take many forms and occur in many places. They include interactions in output markets, input markets and various institutions. Some of these interactions can be illustrated in a stylised diagrammatic representation of an economy known as a “circular flow of economic activity” diagram. These “circular flow” diagrams can incorporate varying levels of detail.

\sphinxAtStartPar
The next four slides contain some examples for a closed economy. While not being a formal economic model itself, a sufficiently detailed circular flow diagram can provide a very useful guide to the construction of a formal general equilibrium economic model of an economy.


\subsection{Circular flow diagram examples}
\label{\detokenize{01.intro_to_economics:circular-flow-diagram-examples}}
\begin{figure}[htbp]
\centering
\capstart

\noindent\sphinxincludegraphics[width=0.800\linewidth]{{circular_flow1}.png}
\caption{Circular flow example 1: households and firms}\label{\detokenize{01.intro_to_economics:id5}}\end{figure}

\begin{figure}[htbp]
\centering
\capstart

\noindent\sphinxincludegraphics[width=0.800\linewidth]{{circular_flow2}.png}
\caption{Circular flow example 2: adding government}\label{\detokenize{01.intro_to_economics:id6}}\end{figure}

\begin{figure}[htbp]
\centering
\capstart

\noindent\sphinxincludegraphics[width=0.800\linewidth]{{circular_flow3}.png}
\caption{Circular flow example 3: households, firms and financial markets}\label{\detokenize{01.intro_to_economics:id7}}\end{figure}

\begin{figure}[htbp]
\centering
\capstart

\noindent\sphinxincludegraphics[width=0.800\linewidth]{{circular_flow4}.png}
\caption{Circular flow example 4: households, firms, financial markets and government}\label{\detokenize{01.intro_to_economics:id8}}\end{figure}


\subsection{The three main components of mathematical (micro\sphinxhyphen{})economics}
\label{\detokenize{01.intro_to_economics:the-three-main-components-of-mathematical-micro-economics}}
\sphinxAtStartPar
Takashi Kunimoto (Unpublished lecture notes on mathematical economics, 18 May 2010, page 6) notes that, according to Rakesh Vohra (2005, Preface), the three core technical components that underlie much of economic theory are as follows.
\begin{itemize}
\item {} 
\sphinxAtStartPar
Feasibility questions

\item {} 
\sphinxAtStartPar
Optimality questions

\item {} 
\sphinxAtStartPar
Equilibrium (fixed\sphinxhyphen{}point) questions

\end{itemize}

\sphinxAtStartPar
This is consistent with the earlier discussion in these lecture slides. One of the main tasks of mathematical (micro\sphinxhyphen{}) economics is the provision of techniques to represent, analyse, and answer these three questions (and various other related questions).


\subsection{Optimisation in economics: motivational quotes}
\label{\detokenize{01.intro_to_economics:optimisation-in-economics-motivational-quotes}}\begin{quote}

\sphinxAtStartPar
“For since the fabric of the universe is most perfect and the work of a most wise Creator, nothing at all takes place in the universe in which some rule of maximum or minimum does not appear.”
\end{quote}

\sphinxAtStartPar
– Leonhard Euler in the introduction to \sphinxstyleemphasis{De Curvis Elasticis}, Additamentum 1 to \sphinxstyleemphasis{Methodus inveniendi lineas curvas maximi minimive proprietate gaudentes, sive Solutio problematis isoperimetrici latissimo sensu accepti} (1744).


\begin{quote}

\sphinxAtStartPar
“Constrained\sphinxhyphen{}maximization problems are mother’s milk to the well\sphinxhyphen{}trained economist.”
\end{quote}

\sphinxAtStartPar
– From page 88 of Caves, Richard E (1980), “Industrial organisation, corporate strategy and structure”, \sphinxstyleemphasis{The Journal of Economic Literature 18(1)}, March, pp. 64–92.


\begin{quote}

\sphinxAtStartPar
“Almost every economic problem involves the study of an agent’s optimal choice as a function of certain parameters or state variables. For example, demand theory is concerned with an agent’s optimal consumption as a function of prices and income, while capital theory studies the optimal investment rule as a function of the existing capital stock.”
\end{quote}

\sphinxAtStartPar
– From page 99 of Ausubel, LM, and RJ Deneckere (1993), “A generalized theorem of the maximum”, \sphinxstyleemphasis{Economic Theory 3(1)}, January, pp. 99–107.


\begin{quote}

\sphinxAtStartPar
“The very name of my subject, economics, suggests economizing or maximising. · · · So at the very foundations of our subject maximization is involved.”
\end{quote}

\sphinxAtStartPar
– From page 249 of Samuelson, (1972), “Maximum principles in analytical economics”, \sphinxstyleemphasis{The American Economic Review 62(3)}, June, pp. 249–262. This journal article is the text of Paul Samuelson’s Nobel Memorial Prize Lecture from 11 November 1970.

\sphinxstepscope


\chapter{Sets, numbers, coordinates, and distances}
\label{\detokenize{02.sets_numbers_coordinates_distances:sets-numbers-coordinates-and-distances}}\label{\detokenize{02.sets_numbers_coordinates_distances::doc}}

\section{Reading guide}
\label{\detokenize{02.sets_numbers_coordinates_distances:reading-guide}}
\sphinxAtStartPar
Introductory level references:
\begin{itemize}
\item {} 
\sphinxAtStartPar
Haeussler, EF Jr, and RS Paul (1987), \sphinxstyleemphasis{Introductory mathematical analysis for business, economics, and the life and social sciences (fifth edition)}, Prentice\sphinxhyphen{}Hall, USA:Chapter 0.2 (pp. 1–3).

\item {} 
\sphinxAtStartPar
Sydsaeter, K, P Hammond, A Strom, and A Carvajal (2016), \sphinxstyleemphasis{Essential mathematics for economic analysis (fifth edition)}, Pearson Education, Italy: Chapters 1.1, 2.1, and 5.5 (pp. 1–12, 19–22, and 160–163).

\item {} 
\sphinxAtStartPar
Shannon, J (1995), \sphinxstyleemphasis{Mathematics for business, economics and finance}, John Wiley and Sons, Brisbane: Chapter 1.2 and 1,3 (pp.2–11).

\end{itemize}

\sphinxAtStartPar
More advanced references:
\begin{itemize}
\item {} 
\sphinxAtStartPar
Banks, J, G Elton and J Strantzen (2009), \sphinxstyleemphasis{Topology and analysis: Unit text for MAT3TA (2009 and 2010 edition)}, Department of Mathematics and Statistics, La Trobe University, Bundoora, February.

\item {} 
\sphinxAtStartPar
Corbae, D, MB Stinchcombe and J Zeman (2009), \sphinxstyleemphasis{An introduction to mathematical analysis for economic theory and econometrics}, Princeton University Press, USA: Chapters 1 and 2 (pp. 1\sphinxhyphen{}71).

\item {} 
\sphinxAtStartPar
Halmos, PR (1960), \sphinxstyleemphasis{Naive set theory}, The University Series in Undergraduate Mathematics, D Van Nostrand Company, USA.

\item {} 
\sphinxAtStartPar
Kolmogorov, AN and SV Fomin (1970), \sphinxstyleemphasis{Introductory real analysis}, Translated and Edited by RA Silverman, The 1975 Dover Edition (an unabridged, slightly corrected republication of the original 1970 Prentice\sphinxhyphen{}Hall edition), Dover Publications, USA: Chapter 1 (pp. 1\sphinxhyphen{}36).

\item {} 
\sphinxAtStartPar
Simon, C, and L Blume (1994), \sphinxstyleemphasis{Mathematics for economists}, WW Norton and Co, USA: Appendix A1 (pp. 847\sphinxhyphen{}858).

\end{itemize}


\section{Sets and elements}
\label{\detokenize{02.sets_numbers_coordinates_distances:sets-and-elements}}
\sphinxAtStartPar
A \sphinxstylestrong{set} (\(X\)) is a collection of objects. A particular object within a set (\(x\)) is known as an \sphinxstylestrong{element} of that set. The idea that x is an element of X is written in mathematical notation as \(x \in X\).

\sphinxAtStartPar
Suppose that there are elements that do not belong to \(X\). Let \(y\) be one such element. The idea that \(y\) is not an element of \(X\) is written in mathematical
notation as \(y \notin X\) .

\sphinxAtStartPar
A set that does not contain any elements is said to be empty. An \sphinxstylestrong{empty set} is denoted by either \(\varnothing\) or \(\{\}\).

\sphinxAtStartPar
There are two fundamental ways of defining a particular set:
\begin{itemize}
\item {} 
\sphinxAtStartPar
The first of these is to exhaustively list all of the elements of the set.
\begin{itemize}
\item {} 
\sphinxAtStartPar
Example 1: \(X = \{1, 2, 3\}\)

\item {} 
\sphinxAtStartPar
Example 2: \(Y = \{1, 2, 3, · · · , 100\}\)

\item {} 
\sphinxAtStartPar
Example 3: \(\mathbb{N} = \{1, 2, 3, · · · \}\)

\end{itemize}

\item {} 
\sphinxAtStartPar
The second of these is to specify one or more properties that characterise all of the elements in the set.
\begin{itemize}
\item {} 
\sphinxAtStartPar
Example 4: \(X = \{n \in \mathbb{N} : n < 4\}\)

\item {} 
\sphinxAtStartPar
Example 5: \(Y = \{n \in \mathbb{N} : n \leqslant 100\}\)

\end{itemize}

\end{itemize}


\subsection{Russell’s Paradox}
\label{\detokenize{02.sets_numbers_coordinates_distances:russell-s-paradox}}
\sphinxAtStartPar
It would be nice if we could always associate some type of set with any particular property that we might consider. In other words, it would be nice if for any property \(\mathbb{A}\), we could form a set \(\{x \in X : \mathbb{A}(x) \text{ is true}\}\) that consisted of all of the elements that satisfy this property. Unfortunately, this is not the case.

\sphinxAtStartPar
This was established by Bertrand Russell. He did this by developing the following paradox.

\sphinxAtStartPar
Let \(\mathbb{A}\) be the property “is a set and does not belong to itself”. Suppose that \(A\) is the set of all sets that possess property \(\mathbb{A}\). Is \(A \in A\)?

\sphinxAtStartPar
If \(A \in A\), then it must be the case that \(A\) possesses property \(\mathbb{A}\). This means that \(A \notin A\). Contradiction! Thus it must be the case that \(A \notin A\). But if \(A\) is a set and \(A \notin A\), then it clearly possesses property \(\mathbb{A}\). Thus \(A \in A\). Contradiction. Thus it must be the case that \(A \in A\).

\sphinxAtStartPar
We have a paradox. It cannot be the case that both \(A \in A\) and \(A \notin A\).

\sphinxAtStartPar
One possible resolution to Russell’s paradox is to not allow mathematical objects like this particular \(A\) to be considered a set.


\subsection{Some common number sets}
\label{\detokenize{02.sets_numbers_coordinates_distances:some-common-number-sets}}\begin{itemize}
\item {} 
\sphinxAtStartPar
The set of natural numbers: \(\mathbb{N} = \{1, 2, 3, · · · \}\)

\item {} 
\sphinxAtStartPar
The set of non\sphinxhyphen{}negative intergers: \(\mathbb{Z}_+ = \{0, 1, 2, · · · \}\)

\item {} 
\sphinxAtStartPar
The set of integers: \(\mathbb{Z} = \{· · · , −2, −1, 0, 1, 2, · · · \}\)

\item {} 
\sphinxAtStartPar
The set of rational numbers: \(\mathbb{Q} = \{ \frac{m}{n} : m \in \mathbb{Z}, n \in \mathbb{N}\}\);
\begin{itemize}
\item {} 
\sphinxAtStartPar
The set of non\sphinxhyphen{}negative rational numbers: \(Q_+ = \{x \in \mathbb{Q} : x \geqslant 0\}\)

\item {} 
\sphinxAtStartPar
The set of positive rational numbers: \(Q_{++} = \{x \in \mathbb{Q} : x > 0\}\);

\end{itemize}

\item {} 
\sphinxAtStartPar
The set of real numbers: \(\mathbb{R} = (−∞, ∞)\)
\begin{itemize}
\item {} 
\sphinxAtStartPar
The set of non\sphinxhyphen{}negative real numbers: \(\mathbb{R}_+ = \{x \in \mathbb{R} : x \geqslant 0\}\)

\item {} 
\sphinxAtStartPar
The set of positive real numbers: \(\mathbb{R}_{++} = \{x \in \mathbb{R}: x > 0\}\)

\end{itemize}

\item {} 
\sphinxAtStartPar
The set of complex numbers:

\end{itemize}
\begin{equation*}
\begin{split}\mathbb{C} = \{ a + bi : a \in \mathbb{R}, \: b \in \mathbb{R}, \: i = \sqrt{−1} \}\end{split}
\end{equation*}

\subsection{Subsets}
\label{\detokenize{02.sets_numbers_coordinates_distances:subsets}}
\sphinxAtStartPar
Consider two sets, \(X\) and \(Y\).

\sphinxAtStartPar
Suppose that every element of \(X\) also belongs to \(Y\). If this is the case, then we say that \(X\) is a \sphinxstylestrong{subset} of \(Y\). This is written in mathematical notation as \(X \subseteq Y\).

\sphinxAtStartPar
Suppose that in addition to every element of \(X\) also belonging to \(Y\), there is at least one element of Y that does not belong to \(X\). If this is the case, then we say that \(X\) is a \sphinxstylestrong{proper subset} of \(Y\). This is written is mathematical notation as \(X \subset Y\).

\sphinxAtStartPar
Sometimes \(X \subset Y\) is used (rather loosely) to mean \(X \subseteq Y\). If this meaning of the notation is employed, then \(X \subsetneq Y\) would need to be used to indicate that \(X\) is a proper subset of \(Y\).

\sphinxAtStartPar
Suppose that both every element of \(X\) also belongs to \(Y\), and every element of \(Y\) also belongs to \(X\). If this is the case, then we say that \(X\) is equal to \(Y\). This is written in mathematical notation as \(X = Y\).

\begin{sphinxadmonition}{note}{Note:}
\sphinxAtStartPar
Recall the common number sets from above. The following “nesting” relationship exists between these common sets of numbers:
\begin{equation*}
\begin{split}\mathbb{N} \subset \mathbb{Z}_+ \subset \mathbb{Z} \subset \mathbb{Q} \subset \mathbb{R} \subset \mathbb{C}\end{split}
\end{equation*}\end{sphinxadmonition}


\subsection{The nesting of number sets}
\label{\detokenize{02.sets_numbers_coordinates_distances:the-nesting-of-number-sets}}
\sphinxAtStartPar
Note that \(\mathbb{N} \subset \mathbb{Z}_+\) because \(\mathbb{Z}_+ = \mathbb{N} \cup \{0\}\).

\sphinxAtStartPar
Note that \(\mathbb{Z}_+ \subset \mathbb{Z}\) because \(\mathbb{Z} = \mathbb{Z}_+ \cup \{ · · · , −3, −2, −1 \}\).

\sphinxAtStartPar
Note that \(\mathbb{Z} \subset \mathbb{Q}\) because any \(m \in \mathbb{Z}\) can be written as \(\frac{m}{1}\) and \(1 \in \mathbb{N}\), but there are fractions that do not belong to Z (for example \(\frac{1}{2} \notin \mathbb{Z}\)).

\sphinxAtStartPar
Note that \(\mathbb{Q} \subset \mathbb{R}\) because \(\frac{m}{n} \in (−\infty, \infty)\) for all \(m \in \mathbb{Z}\) and \(n \in \mathbb{N}\), but there are numbers on the real line that cannot be expressed as fractions (for example \(\sqrt{2}\), \(\pi\) and \(e\)). Real numbers that cannot be expressed as fractions are known as “irrational numbers”.

\sphinxAtStartPar
Note that \(\mathbb{R} \subset \mathbb{C}\) because
\begin{equation*}
\begin{split}\mathbb{R} = \{ a + bi : a \in \mathbb{R}, \: b = 0, \: i = \sqrt{−1} \}\end{split}
\end{equation*}
\sphinxAtStartPar
and \(0 \in \mathbb{R}\), but \((a + bi) \notin \mathbb{R}\) if \(b \neq 0\).
\begin{itemize}
\item {} 
\sphinxAtStartPar
Complex numbers in which a = 0 are known as (purely) “imaginary numbers”.

\end{itemize}


\subsection{Intervals as subsets of the real line}
\label{\detokenize{02.sets_numbers_coordinates_distances:intervals-as-subsets-of-the-real-line}}
\sphinxAtStartPar
Some (but not all) of the subsets of the real line take the form of an
interval. There are four types of interval. Let \(a \in \mathbb{R}, b \in \mathbb{R} \text{ and } a < b\). The four types of interval are as follows:
\begin{itemize}
\item {} 
\sphinxAtStartPar
\([a, b] = {x \in \mathbb{R} : a \leqslant x \leqslant b}\). (If \(a > b\), then \([a, b] = \varnothing\).) (If \(a = b\), then \([a, b] = \{a\} = \{b\}\).)

\item {} 
\sphinxAtStartPar
\((a, b) = {x \in \mathbb{R} : a < x < b}\). (If \(a > b\), then \([a, b] = \varnothing\).)

\item {} 
\sphinxAtStartPar
\([a, b) = {x \in \mathbb{R} : a \leqslant x < b}\). (If \(a > b\), then \([a, b] = \varnothing\).)

\item {} 
\sphinxAtStartPar
\((a, b] = {x \in \mathbb{R} : a < x \leqslant b}\). (If \(a > b\), then \([a, b] = \varnothing\).)

\end{itemize}


\section{The real number system}
\label{\detokenize{02.sets_numbers_coordinates_distances:the-real-number-system}}
\sphinxAtStartPar
The system of real numbers is an algebraic structure known as a \sphinxstylestrong{complete ordered field}. Indeed, in a sense, it is the only complete ordered field in existence. Any other complete ordered field turns out to be “isomoporhic” to the real number system. (The term “isomorphic” is a mathematical version of “essentially the same”).

\sphinxAtStartPar
The system of real numbers is formally denoted by \((\mathbb{R}, \mathbb{R}_{++}, +, \times)\), where \(+\) and \(\times\) are the familiar addition and multiplication operations for real numbers.

\sphinxAtStartPar
The set of real numbers \(\mathbb{R}\) can be viewed as the completion of the the set of rational numbers \(\mathbb{Q}\) because it involves filling in the “holes” that exist in the set of rational numbers. These holes take the form of irrational numbers like \(\sqrt{2}\), \(\pi\) and \(e\).


\subsection{Why aren’t the rationals enough?}
\label{\detokenize{02.sets_numbers_coordinates_distances:why-aren-t-the-rationals-enough}}
\sphinxAtStartPar
Why aren’t the rational numbers enough? What makes us think that they contain “holes”? This is a very good question. Especially when you realise that any numerical calculation on a computer will only generate a rational number.

\sphinxAtStartPar
A geometric argument for the existence of irrational numbers is perhaps the easiest way to convince yourself of their existence. Think about a right\sphinxhyphen{}angled triangle with a base (horizontal side) that is one metre long and whose (perpendicular) height (which is also its vertical side) is also one metre long. We know from Pythagoras’ Theorem that the length of the hypotenuse for this triangle is “the
square root of two” metres long. But it can be shown that \(\sqrt{2}\) is an irrational number!


\subsection{Algebraic rules for the real number system}
\label{\detokenize{02.sets_numbers_coordinates_distances:algebraic-rules-for-the-real-number-system}}
\sphinxAtStartPar
Consider any three real numbers: \(a \in \mathbb{R}\), \(b \in \mathbb{R}\), and \(c \in \mathbb{R}\). Let \(0 \in \mathbb{R}\) be the additive identity element and \(1 \in \mathbb{R}\) be the multiplicative identity element. The real numbers obey the following algebraic rules:
\begin{itemize}
\item {} 
\sphinxAtStartPar
Commutative Law for Addition: \((a + b) = (b + a)\).

\item {} 
\sphinxAtStartPar
Associative Law for Addition: \(a + (b + c) = (a + b) + c\).

\item {} 
\sphinxAtStartPar
Existence of the Additive Identity (\(0\)): \(a + 0 = a\).

\item {} 
\sphinxAtStartPar
Existence of an Additive Inverse (\(−a\)): \(a + (−a) = 0\).

\item {} 
\sphinxAtStartPar
Commutative Law for Multiplication: \(a \times b = b \times a\).

\item {} 
\sphinxAtStartPar
Associative Law for Multiplication: \(a \times (b \times c) = (a \times b) \times c\).

\item {} 
\sphinxAtStartPar
Existence of the Multiplicative Identity (\(1\)): \(a \times 1 = a\).

\item {} 
\sphinxAtStartPar
Existence of a Multiplicative Inverse (\(a^{−1}\)): \(a \times a^{−1} = 1\) for all \(a \ne 0\).

\item {} 
\sphinxAtStartPar
Left Distributive Law: \(a \times (b + c) = a \times b + a \times c\).

\item {} 
\sphinxAtStartPar
Right Distributive Law: \((a + b) \times c = a \times c + b \times c\).

\end{itemize}

\sphinxAtStartPar
The real numbers also possess the Archimedean property. This simply says that for any \(x \in \mathbb{R}\), there exists some \(n \in \mathbb{N}\) such that \(x < n\).

\sphinxAtStartPar
The real numbers also obey the following algebraic rules.
\begin{itemize}
\item {} 
\sphinxAtStartPar
Multiplication by Zero: \(a \times 0 = 0\).

\item {} 
\sphinxAtStartPar
Non\sphinxhyphen{}Existence of a Multiplicative Inverse for Zero: \(0^{−1} = \frac{1}{0}\) is undefined.

\item {} 
\sphinxAtStartPar
Multiplication of a Positive by a Negative: \(a \times (−b) = −(a \times b)\).

\item {} 
\sphinxAtStartPar
Multiplication of a Negative by a Negative: \((−a) \times (−b) = (a \times b)\).

\item {} 
\sphinxAtStartPar
Multiplication of Inequalities by (\(−1\)).
\begin{itemize}
\item {} 
\sphinxAtStartPar
\(a \leqslant b \iff (−a) \geqslant (−b)\).

\item {} 
\sphinxAtStartPar
\(a < b \iff (−a) > (−b)\).

\item {} 
\sphinxAtStartPar
\(a > b \iff (−a) < (−b)\).

\item {} 
\sphinxAtStartPar
\(a \geqslant b \iff (−a) \leqslant (−b)\).

\end{itemize}

\item {} 
\sphinxAtStartPar
Order Reversal for Multiplicative Inverses (Fractions).
\begin{itemize}
\item {} 
\sphinxAtStartPar
Assume that both \(a \ne 0\) and \(b \ne 0\) in the following two statements.

\item {} 
\sphinxAtStartPar
\(a \leqslant b \iff a^{−1} > b^{−1}\) (that is, \(\frac{1}{a} \geqslant \frac{1}{b}\) ).

\item {} 
\sphinxAtStartPar
\(a < b \iff a^{−1} > b^{−1}\) (that is, \(\frac{1}{a} > \frac{1}{b}\)).

\end{itemize}

\end{itemize}


\section{More on sets}
\label{\detokenize{02.sets_numbers_coordinates_distances:more-on-sets}}

\subsection{Power sets}
\label{\detokenize{02.sets_numbers_coordinates_distances:power-sets}}
\sphinxAtStartPar
The \sphinxstylestrong{power set} (\(2^X\)) of a set (\(X\)) is the set of all subsets of that set. Note that the elements of a power set are sets themselves.

\sphinxAtStartPar
If there are \(n < \infty\) elements (that is, a finite number of elements) in the set \(X\), then the number of subsets of \(X\) will be equal to \(2^n\). As such, there will be \(2^n\) elements in the set \(2^X\). This might be the reason that the power set for some set \(X\) is often denoted by \(2^X\).

\sphinxAtStartPar
Example: Suppose that \(X = \{1, 2, 3\}\). The power set for the set \(X\) in this example is
\begin{equation*}
\begin{split}2^X = \{\varnothing, \{1\} , \{2\} , \{3\} , \{1, 2\} , \{1, 3\} , \{2, 3\} , \{1, 2, 3\}\}\end{split}
\end{equation*}
\sphinxAtStartPar
Note that there are three elements in the set \(X\) and eight elements in the power set for \(X\). Note also that \(2^3 = 8\).


\subsection{Cartesian products}
\label{\detokenize{02.sets_numbers_coordinates_distances:cartesian-products}}
\sphinxAtStartPar
The \sphinxstylestrong{Cartesian product} of \sphinxstyleemphasis{two} sets is defined to be the set of all ordered \sphinxstyleemphasis{pairs} (or doublets) that contain one component from each of the two sets in the order that the sets were specified. This can be formally expressed as
\begin{equation*}
\begin{split}X \times Y = \{(x, y) : x \in X , \: y \in Y \}\end{split}
\end{equation*}
\sphinxAtStartPar
The Cartesian product of \(n\) sets is defined to be the set of all ordered \(n\)\sphinxhyphen{}tuples that contain one component from each of the \(n\) sets in the order that the sets were specified. This can be formally expressed as
\begin{equation*}
\begin{split}\times_{i \in \{1,2,··· ,n\}} X_i = \{(x_1, x_2, · · · , x_n) : x_i \in X_i \; \forall i \in \{1, 2, · · · , n\}\}\end{split}
\end{equation*}
\sphinxAtStartPar
Note that the order of the sets matters here. Cartesian products generate sets of “ordered” n\sphinxhyphen{}tuples.


\subsubsection{Examples}
\label{\detokenize{02.sets_numbers_coordinates_distances:examples}}\begin{itemize}
\item {} 
\sphinxAtStartPar
The standard two\sphinxhyphen{}dimensional Euclidean coordinate plane from high school:

\end{itemize}
\begin{equation*}
\begin{split}\mathbb{R}^2 = \mathbb{R} \times \mathbb{R} = \{(x, y) : x \in \mathbb{R}, \; y \in \mathbb{R}\}\end{split}
\end{equation*}\begin{itemize}
\item {} 
\sphinxAtStartPar
The \(n\)\sphinxhyphen{}dimensional Euclidean coordinate plane:

\end{itemize}
\begin{equation*}
\begin{split}\mathbb{R}^n = \times_{i \in \{1,2,··· ,n\}} \mathbb{R} = \{(x_1, x_2, · · · , x_n) : x_i \in \mathbb{R} \; \forall i \in \{1, 2, · · · , n\}\}\end{split}
\end{equation*}\begin{itemize}
\item {} 
\sphinxAtStartPar
A discrete\sphinxhyphen{}continuous example:

\end{itemize}
\begin{equation*}
\begin{split}\mathbb{N} \times \mathbb{R} = \{(n, y) : n \in \mathbb{N}, \; y \in \mathbb{R}\}\end{split}
\end{equation*}\begin{itemize}
\item {} 
\sphinxAtStartPar
A continuous\sphinxhyphen{}discrete example:

\end{itemize}
\begin{equation*}
\begin{split}\mathbb{R} \times \mathbb{N} = \{(x, n) : x \in \mathbb{R}, \; n \in \mathbb{N}\}\end{split}
\end{equation*}\begin{itemize}
\item {} 
\sphinxAtStartPar
If X = \{1, 2, 3\}, then the Cartesian product of X with itself is given by

\end{itemize}
\begin{equation*}
\begin{split}X^2 = X \times X = \{(x, y) : x \in X , \; y \in X \}\end{split}
\end{equation*}\begin{itemize}
\item {} 
\sphinxAtStartPar
This set can also be written out as an exhaustive list of possible cases as follows:

\end{itemize}
\begin{equation*}
\begin{split}X^2 = \{(1, 1) , (1, 2) , (1, 3) , (2, 1), (2, 2) , (2, 3) , (3, 1) , (3, 2) , (3, 3)\}\end{split}
\end{equation*}

\subsection{Non\sphinxhyphen{}negative and positive real orthants}
\label{\detokenize{02.sets_numbers_coordinates_distances:non-negative-and-positive-real-orthants}}\begin{itemize}
\item {} 
\sphinxAtStartPar
The set of non\sphinxhyphen{}negative real numbers is given by

\end{itemize}
\begin{equation*}
\begin{split}R_+ = \{x \in \mathbb{R} : x \geqslant 0\} = [0, \infty)\end{split}
\end{equation*}\begin{itemize}
\item {} 
\sphinxAtStartPar
The Euclidean \(L\)\sphinxhyphen{}dimensional non\sphinxhyphen{}negative orthant is

\end{itemize}
\begin{equation*}
\begin{split}R^L_+ = \times_{I = 1}^L \mathbb{R}_+\end{split}
\end{equation*}\begin{itemize}
\item {} 
\sphinxAtStartPar
The set of positive real numbers is given by

\end{itemize}
\begin{equation*}
\begin{split}R_{++} = \{x \in \mathbb{R} : x > 0\} = (0, \infty)\end{split}
\end{equation*}\begin{itemize}
\item {} 
\sphinxAtStartPar
The Euclidean \(L\)\sphinxhyphen{}dimensional positive orthant is

\end{itemize}
\begin{equation*}
\begin{split}R^L_{++} = \times_{I = 1}^L \mathbb{R}_{++}\end{split}
\end{equation*}

\subsection{Non\sphinxhyphen{}positive and negative real orthants}
\label{\detokenize{02.sets_numbers_coordinates_distances:non-positive-and-negative-real-orthants}}\begin{itemize}
\item {} 
\sphinxAtStartPar
The set of non\sphinxhyphen{}positive real numbers is given by

\end{itemize}
\begin{equation*}
\begin{split}R_- = \{x \in \mathbb{R} : x \leqslant 0\} = (-\infty, 0]\end{split}
\end{equation*}\begin{itemize}
\item {} 
\sphinxAtStartPar
The Euclidean \(L\)\sphinxhyphen{}dimensional non\sphinxhyphen{}positive orthant is

\end{itemize}
\begin{equation*}
\begin{split}R^L_- = \times_{I = 1}^L \mathbb{R}_-\end{split}
\end{equation*}\begin{itemize}
\item {} 
\sphinxAtStartPar
The set of negative real numbers is given by

\end{itemize}
\begin{equation*}
\begin{split}R_{--} = \{x \in \mathbb{R} : x < 0\} = (-\infty, 0)\end{split}
\end{equation*}\begin{itemize}
\item {} 
\sphinxAtStartPar
The Euclidean \(L\)\sphinxhyphen{}dimensional negative orthant is

\end{itemize}
\begin{equation*}
\begin{split}R^L_{--} = \times_{I = 1}^L \mathbb{R}_{--}\end{split}
\end{equation*}

\section{Distances in Euclidean spaces}
\label{\detokenize{02.sets_numbers_coordinates_distances:distances-in-euclidean-spaces}}
\sphinxAtStartPar
Suppose that \(x, y \in \mathbb{R}^n\), where \(n \in \mathbb{N}\). The coordinates of the two points will take the form \(x = (x_1, x_2, · · · , x_n)\) and \(y = (y_1, y_2, · · · , y_n)\).

\sphinxAtStartPar
The Euclidean distance between these two points is given by the Euclidean distance metric:
\begin{equation*}
\begin{split}d (x, y) = \sqrt{\sum_{i = 1}^{n} (y_i - x_i)^2}\end{split}
\end{equation*}
\sphinxAtStartPar
Note that when \(n = 1\), so that \(\mathbb{R}^n = \mathbb{R}\), we have
\begin{equation*}
\begin{split}d (x, y) = \sqrt{(y - x)^2} = \lvert(y - x)\rvert\end{split}
\end{equation*}

\subsection{Properties of distance metrics}
\label{\detokenize{02.sets_numbers_coordinates_distances:properties-of-distance-metrics}}
\sphinxAtStartPar
The Euclidean distance metric is just one example of many possible distance metrics on sets of the form \(\mathbb{R}^n\) for some \(n \in \mathbb{N}\). It is the metric that is most commonly used for these sets in economics (and, I suspect, a number of other disciplines). But it is not the only possible distance metric.

\sphinxAtStartPar
All distance metrics must satisfy a number of properties to ensure that they are valid measures of distance. To be precise, any distance metric \(d(x, y)\) on a set \(S\) must satisfy each of the following properties.
\begin{itemize}
\item {} 
\sphinxAtStartPar
(DM1: Non\sphinxhyphen{}Negativity): \(d(x, y) \geqslant 0\) for all \(x, y \in S\).

\item {} 
\sphinxAtStartPar
(DM2: Separability): \(d(x, y) = 0\) if and only if \(y = x\).

\item {} 
\sphinxAtStartPar
(DM3: Symmetry): \(d(x, y) = d(y , x)\) for all \(x, y \in S\).

\item {} 
\sphinxAtStartPar
(DM4: The Triangle Inequality): \(d (x, y ) \leqslant d (x, z) + d (z, y)\) for all \(x, y, z \in S\).

\end{itemize}


\section{Operations on sets}
\label{\detokenize{02.sets_numbers_coordinates_distances:operations-on-sets}}

\subsection{Union and intersection}
\label{\detokenize{02.sets_numbers_coordinates_distances:union-and-intersection}}
\sphinxAtStartPar
Suppose that U is some universal set, \(X \subseteq U\) and \(Y \subseteq U\).

\sphinxAtStartPar
The \sphinxstylestrong{union} of \(X\) and \(Y\), which is denoted by \(X \cup Y\), is the set
\begin{equation*}
\begin{split}X \cup Y = \{a \in U : a \in X \text{ or } a \in Y\}\end{split}
\end{equation*}
\sphinxAtStartPar
Note that the “or” in this definition is not exclusive. If the element \(a\) belongs to either \(X\) only, or \(Y\) only, or both \(X\) and \(Y\) , then \(a \in X \cup Y\).

\sphinxAtStartPar
The \sphinxstylestrong{intersection} of \(X\) and \(Y\), which is denoted by \(X \cap Y\), is the set
\begin{equation*}
\begin{split}X \cap Y = \{a \in U : a \in X \text{ and } a \in Y\}\end{split}
\end{equation*}
\sphinxAtStartPar
If \(X \cap Y = \varnothing\), then the sets \(X\) and \(Y\) are said to be \sphinxstylestrong{disjoint}.

\sphinxAtStartPar
\sphinxstyleemphasis{Illustrate set union and set intersection on the whiteboard using Venn diagrams for both the case of disjoint sets and the case of overlapping sets.}


\subsubsection{Examples}
\label{\detokenize{02.sets_numbers_coordinates_distances:id1}}
\begin{figure}[htbp]
\centering
\capstart

\noindent\sphinxincludegraphics[width=0.800\linewidth]{{union_example}.png}
\caption{Examples: union}\label{\detokenize{02.sets_numbers_coordinates_distances:id4}}\end{figure}

\begin{figure}[htbp]
\centering
\capstart

\noindent\sphinxincludegraphics[width=0.800\linewidth]{{intersection_example}.png}
\caption{Examples: intersection}\label{\detokenize{02.sets_numbers_coordinates_distances:id5}}\end{figure}


\subsection{Exclusion and complementation}
\label{\detokenize{02.sets_numbers_coordinates_distances:exclusion-and-complementation}}
\sphinxAtStartPar
Suppose that U is some universal set, \(X \subseteq U\) and \(Y \subseteq U\).

\sphinxAtStartPar
The \sphinxstylestrong{set difference} “X excluding Y”, which is denoted by \(X \setminus Y\), is the set
\begin{equation*}
\begin{split}X \setminus Y = \{a \in X : a \notin Y \}\end{split}
\end{equation*}
\sphinxAtStartPar
Set complementation is a special case of set exclusion. The \sphinxstylestrong{complement} of the set \(X\), which is denoted by \(X^C\) , is defined as \(X^C = U \setminus X\).
\begin{itemize}
\item {} 
\sphinxAtStartPar
Note that \(X \cup X^C = U\) and \(X \cap X^C = \varnothing\).

\end{itemize}

\sphinxAtStartPar
\sphinxstyleemphasis{Illustrate set exclusion and set complementation on the whiteboard using Venn diagrams for both the case of disjoint sets and the case of overlapping sets.}


\subsubsection{Examples}
\label{\detokenize{02.sets_numbers_coordinates_distances:id2}}
\begin{figure}[htbp]
\centering
\capstart

\noindent\sphinxincludegraphics[width=0.800\linewidth]{{exclusion_example}.png}
\caption{Examples: exclusion/set difference}\label{\detokenize{02.sets_numbers_coordinates_distances:id6}}\end{figure}

\begin{sphinxadmonition}{note}{Example: Set complementation}

\sphinxAtStartPar
Suppose that the universal set is \(U = {1, 2, 3}\). The power set for the set \(U\) in this example is
\begin{equation*}
\begin{split}2^U = \{\varnothing, \{1\} , \{2\} , \{3\} , \{1, 2\} , \{1, 3\} , \{2, 3\} , \{1, 2, 3\}\} .\end{split}
\end{equation*}
\sphinxAtStartPar
The elements of this power set are the subsets of the universal set. The complements for each of these subsets are:
\begin{itemize}
\item {} 
\sphinxAtStartPar
\(\varnothing^C = U\)

\item {} 
\sphinxAtStartPar
\(\{1\}^C = \{2, 3\}\)

\item {} 
\sphinxAtStartPar
\(\{2\}^C = \{1, 3\}\)

\item {} 
\sphinxAtStartPar
\(\{3\}^C = \{1, 2\}\)

\item {} 
\sphinxAtStartPar
\(\{1, 2\}^C = \{3\}\)

\item {} 
\sphinxAtStartPar
\(\{1, 3\}^C = \{2\}\)

\item {} 
\sphinxAtStartPar
\(\{2, 3\}^C = \{1\}\)

\item {} 
\sphinxAtStartPar
\(U^C = \varnothing\)

\end{itemize}
\end{sphinxadmonition}


\subsection{Symmetric difference}
\label{\detokenize{02.sets_numbers_coordinates_distances:symmetric-difference}}
\sphinxAtStartPar
The \sphinxstylestrong{symmetric difference} of \(X\) and \(Y\), which is denoted by \(X \bigtriangleup Y\), is
the set
\begin{equation*}
\begin{split}X \bigtriangleup Y = (X \setminus Y) \cup (Y \setminus X)\end{split}
\end{equation*}
\sphinxAtStartPar
It is interesting to compare the operations of union and symmetric difference. They relate to alternative interpretions of the phrase “belongs to either \(X\) or \(Y\)”.
\begin{itemize}
\item {} 
\sphinxAtStartPar
The set \(X \cup Y\) consists of all elements that are either in set \(X\) only, or in set \(Y\) only, or in both of these sets.

\item {} 
\sphinxAtStartPar
The set \(X \bigtriangleup Y\) consists of all elements that are either in set \(X\) only, or in set \(Y\) only, but are not in both of these sets.

\end{itemize}

\sphinxAtStartPar
\sphinxstyleemphasis{Illustrate the symmetric difference on the whiteboard using Venn diagrams for both the case of disjoint sets and the case of overlapping sets.}


\subsubsection{Example}
\label{\detokenize{02.sets_numbers_coordinates_distances:example}}
\begin{figure}[htbp]
\centering
\capstart

\noindent\sphinxincludegraphics[width=0.800\linewidth]{{symmetric_difference_example}.png}
\caption{Examples: symmetric difference}\label{\detokenize{02.sets_numbers_coordinates_distances:id7}}\end{figure}


\section{De Morgan’s laws}
\label{\detokenize{02.sets_numbers_coordinates_distances:de-morgan-s-laws}}
\sphinxAtStartPar
Simple version:

\sphinxAtStartPar
Suppose that \(X\) is some set, \(A \subseteq X\) and \(B \subseteq X\). According to De Morgan’s laws, we have
\begin{equation*}
\begin{split}X \setminus (A \cup B) = (X \setminus A) \cap (X \setminus B) \end{split}
\end{equation*}
\sphinxAtStartPar
and
\begin{equation*}
\begin{split}X \setminus (A \cap B) = (X \setminus A) \cup (X \setminus B) \end{split}
\end{equation*}
\sphinxAtStartPar
General version:

\sphinxAtStartPar
Let \(I\) be some index set. Note that while I is allowed to be finite, it does not have to be finite. Let \(X\) be some set and suppose that \(A_i \subseteq U\) for all \(i \in I\). According to De Morgan’s laws, we have
\begin{equation*}
\begin{split}X \setminus (\cup_{i \in I} A_i) = \cap_{i \in I} (X \setminus A_i) \end{split}
\end{equation*}
\sphinxAtStartPar
and
\begin{equation*}
\begin{split}X \setminus (\cap_{i \in I} A_i) = \cup_{i \in I} (X \setminus A_i) \end{split}
\end{equation*}

\subsection{Example}
\label{\detokenize{02.sets_numbers_coordinates_distances:id3}}
\sphinxAtStartPar
Let \(X = \{1, 2, 3\}, A = \{1, 2\} \subset X\) and \(B = \{2, 3\} \subset X\).  Note that \(A \cup B = \{1, 2, 3\} = X\) and \(A \cap B = \{2\}\). Note also that
\begin{itemize}
\item {} 
\sphinxAtStartPar
\(X \setminus A = \{3\}, X \setminus B = \{1\}\),

\item {} 
\sphinxAtStartPar
\(X \setminus (A \cup B) = X \setminus X = \varnothing\) and

\item {} 
\sphinxAtStartPar
\(X \setminus (A \cap B) = X \setminus \{2\} = \{1, 3\}\).

\end{itemize}

\sphinxAtStartPar
We have
\begin{equation*}
\begin{split}(X \setminus A) \cap (X \setminus B) = \{3\} \cap \{1\} = \varnothing = X \setminus (A \cup B)\end{split}
\end{equation*}
\sphinxAtStartPar
and
\begin{equation*}
\begin{split}(X \setminus A) \cup (X \setminus B) = \{3\} \cup \{1\} = \{1, 3\} = X \setminus (A \cap B)\end{split}
\end{equation*}

\section{Examples of sets in economics}
\label{\detokenize{02.sets_numbers_coordinates_distances:examples-of-sets-in-economics}}
\sphinxAtStartPar
Some sets that you might encounter during your study of economics include:
\begin{itemize}
\item {} 
\sphinxAtStartPar
Budget sets

\item {} 
\sphinxAtStartPar
Weak preference sets

\item {} 
\sphinxAtStartPar
Indifference sets

\item {} 
\sphinxAtStartPar
Input requirement sets

\item {} 
\sphinxAtStartPar
Isoquants

\item {} 
\sphinxAtStartPar
Isocosts

\item {} 
\sphinxAtStartPar
Price simplices (simplexes)

\end{itemize}

\sphinxAtStartPar
We will briefly consider some of these examples below.


\subsection{Budget sets}
\label{\detokenize{02.sets_numbers_coordinates_distances:budget-sets}}\begin{itemize}
\item {} 
\sphinxAtStartPar
Linear prices with an income endowment:

\end{itemize}
\begin{equation*}
\begin{split}B(p_1, p_2, · · · , p_L, y ) = \left\{
(x_1, x_2, · · · , x_L) \in \mathbb{R}_+^L : \sum_{I = 1}^{L} p_I x_I \leqslant y \right\}\end{split}
\end{equation*}\begin{itemize}
\item {} 
\sphinxAtStartPar
Linear prices with a commodity bundle endowment:

\end{itemize}
\begin{equation*}
\begin{split}B(p_1, p_2, · · · , p_L, e_1, e_2, · · · , e_L)
=
\left\{
(x_1, x_2, · · · , x_L) \in \mathbb{R}_+^L : \sum_{I = 1}^L p_I (x_I − e_I ) \leqslant 0
\right\}\end{split}
\end{equation*}
\sphinxAtStartPar
These are examples of “lower contour sets” for expenditure by an individual.

\sphinxAtStartPar
\sphinxstyleemphasis{Illustrate these sets on the whiteboard for the two commodity case.}


\subsection{Weak preference sets}
\label{\detokenize{02.sets_numbers_coordinates_distances:weak-preference-sets}}\begin{itemize}
\item {} 
\sphinxAtStartPar
Reference commodity bundle version:

\end{itemize}
\begin{equation*}
\begin{split}U^+(x_1, x_2, · · · , x_L) =
\left\{
(y_1, y_2, · · · , y_L) \in \mathbb{R}^L_+ : U(y_1, y_2, · · · , y_L) \geqslant U (x_1, x_2, · · · , x_L)
\right\}\end{split}
\end{equation*}\begin{itemize}
\item {} 
\sphinxAtStartPar
Reference utility level version:

\end{itemize}
\begin{equation*}
\begin{split}U^+(k) =
\left\{
(y_1, y_2, · · · , y_L) \in \mathbb{R}^L_+ : U(y_1, y_2, · · · , y_L) \geqslant k
\right\}\end{split}
\end{equation*}
\sphinxAtStartPar
These are examples of “(weak) upper contour sets” for the utility level attained by an individual.

\sphinxAtStartPar
\sphinxstyleemphasis{Illustrate these sets on the whiteboard for the two commodity case.}


\subsection{Indifference sets}
\label{\detokenize{02.sets_numbers_coordinates_distances:indifference-sets}}\begin{itemize}
\item {} 
\sphinxAtStartPar
Reference commodity bundle version:

\end{itemize}
\begin{equation*}
\begin{split}U^0(x_1, x_2, · · · , x_L) =
\left\{
(y_1, y_2, · · · , y_L) \in \mathbb{R}^L_+ : U(y_1, y_2, · · · , y_L) = U(x_1, x_2, · · · , x_L)
\right\}\end{split}
\end{equation*}\begin{itemize}
\item {} 
\sphinxAtStartPar
Reference utility level version:

\end{itemize}
\begin{equation*}
\begin{split}U^0(k) =
\left\{
(y_1, y_2, · · · , y_L) \in \mathbb{R}^L_+ : U(y_1, y_2, · · · , y_L) = k
\right\}\end{split}
\end{equation*}
\sphinxAtStartPar
These are examples of “level sets” for the utility level attained by an individual.

\sphinxAtStartPar
\sphinxstyleemphasis{Illustrate these sets on the whiteboard for the two commodity case.}


\subsection{Input requirement sets}
\label{\detokenize{02.sets_numbers_coordinates_distances:input-requirement-sets}}
\sphinxAtStartPar
Consider a single output, multiple input production technology that can be represented by a production function of the form
\begin{equation*}
\begin{split}y = f(x_1, x_2, · · · , x_n).\end{split}
\end{equation*}
\sphinxAtStartPar
An \sphinxstylestrong{input requirement set} for this production technology takes the form
\begin{equation*}
\begin{split}y^+(k) = 
\left\{(x_1, x_2, · · · , x_n) \in \mathbb{R}^n_+ : f (x_1, x_2, · · · , x_n) \geqslant k
\right\}\end{split}
\end{equation*}
\sphinxAtStartPar
This is an example of a “(weak) upper contour set” for the output level attained by a producer.

\sphinxAtStartPar
\sphinxstyleemphasis{Illustrate this set on the whiteboard for the two input case.}


\subsection{Isoquants}
\label{\detokenize{02.sets_numbers_coordinates_distances:isoquants}}
\sphinxAtStartPar
Consider a single output, multiple input production technology that can be represented by a production function of the form
\begin{equation*}
\begin{split}y = f(x_1, x_2, · · · , x_n).\end{split}
\end{equation*}
\sphinxAtStartPar
An \sphinxstylestrong{isoquant} for this production technology takes the form
\begin{equation*}
\begin{split}y^0(k) = 
\left\{(x_1, x_2, · · · , x_n) \in \mathbb{R}^n_+ : f (x_1, x_2, · · · , x_n) = k 
\right\}\end{split}
\end{equation*}
\sphinxAtStartPar
This is an example of a “level set” or the output level attained by a producer.
\sphinxstyleemphasis{Illustrate this set on the whiteboard for the two input case.}


\subsection{Isocosts}
\label{\detokenize{02.sets_numbers_coordinates_distances:isocosts}}
\sphinxAtStartPar
An \sphinxstylestrong{isocost} depicts the locus of all input combinations that cost the producer the same amount of money to employ.

\sphinxAtStartPar
Suppose that there are \(n \in \mathbb{N}\) distinct production inputs (or factors of
production, if you prefer). An isocost for this situation takes the form
\begin{equation*}
\begin{split}C^0(k) = \left\{(x_1, x_2, · · · , x_n) \in \mathbb{R}^n_+ :
\sum_{i = 1}^{n} w_i x_i = k
\right\}\end{split}
\end{equation*}
\sphinxAtStartPar
where \(w_i\) is the per\sphinxhyphen{}unit price of input \(i\).

\sphinxAtStartPar
This is an example of a “level set” for the expenditure on inputs by a producer.
\sphinxstyleemphasis{Illustrate this set on the whiteboard for the two input case.}


\subsection{Price simplex}
\label{\detokenize{02.sets_numbers_coordinates_distances:price-simplex}}
\sphinxAtStartPar
In some situations in economics, it is relative prices that matter, rather than the absolute level of each individual price. In such cases, some form of price normalisation can be employed.

\sphinxAtStartPar
Common normalisations involve choosing either a particular commodity, or a particular basket of commodities, to be the numeraire. The expenditure on the the numeraire commodity, or numeraire basket of commodities, is then set equal to one.

\sphinxAtStartPar
If the numeraire basket consists of one unit of each of the \(n\) final commodities in an economy, then the set of possible prices is given by:
\begin{equation*}
\begin{split}\bigtriangleup (p_1, p_2, · · · , p_n) =
\left\{ (p_1, p_2, · · · , p_n) \in \mathbb{R}^n_+ :
\sum_{i = 1}^{n} p_i = 1
\right\}\end{split}
\end{equation*}
\sphinxAtStartPar
This set is known as a price “simplex”.

\sphinxAtStartPar
\sphinxstyleemphasis{Illustrate this set on the whiteboard for both the two commodity case and the three commodity case.}

\sphinxstepscope


\chapter{Mappings: functions and correspondences}
\label{\detokenize{03.mappings_functions_correspondences:mappings-functions-and-correspondences}}\label{\detokenize{03.mappings_functions_correspondences::doc}}

\section{Reading guide}
\label{\detokenize{03.mappings_functions_correspondences:reading-guide}}
\sphinxAtStartPar
Introductory mathematical economics references:
\begin{itemize}
\item {} 
\sphinxAtStartPar
Haeussler, EF Jr, and RS Paul (1987), \sphinxstyleemphasis{Introductory mathematical analysis for business, economics, and the life and social sciences (fifth edition)}, Prentice\sphinxhyphen{}Hall, USA:Chapters 0, 3, 4, 5, and 17.1.

\item {} 
\sphinxAtStartPar
Sydsaeter, K, P Hammond, A Strom, and A Carvajal (2016), \sphinxstyleemphasis{Essential mathematics for economic analysis (fifth edition)}, Pearson Education, Italy: Chapters 2, 4, 5, 9.6 (pp. 350\sphinxhyphen{}351 only), 11.1, and 11.5.

\item {} 
\sphinxAtStartPar
Shannon, J (1995), \sphinxstyleemphasis{Mathematics for business, economics and finance}, John Wiley and Sons, Brisbane: Chapters 1, 2, and 6.

\end{itemize}

\sphinxAtStartPar
Advanced high school references:
\begin{itemize}
\item {} 
\sphinxAtStartPar
Coroneos, J. (Undated a), \sphinxstyleemphasis{A Higher School Certificate Course in Mathematics: Year Eleven, Three Unit Course}, Coroneos Publications, Australia: Chapters 1, 2, 4, and 7.

\item {} 
\sphinxAtStartPar
Coroneos, J. (Undated b), \sphinxstyleemphasis{A Higher School Certificate Course in Mathematics: Years Eleven and Twelve, Revised Four Unit Course (for Mathematics Extension Two)}, Coroneos Publications, Australia: Chapters 1 and 2.

\end{itemize}

\sphinxAtStartPar
Introductory mathematics references:
\begin{itemize}
\item {} 
\sphinxAtStartPar
Adams, RA, and C Essex (2018), \sphinxstyleemphasis{Calculus: A complete course (ninth edition)} Pearson, Canada: Chapters P, 3, and 11.

\item {} 
\sphinxAtStartPar
Kline, M (1967), \sphinxstyleemphasis{Calculus: An intuitive and physical approach (second edition)}, The 1998 Dover republication of the original John Wiley and Sons second edition, Dover Publications, USA: pp. 419–432.

\item {} 
\sphinxAtStartPar
Silverman, RA (1969), \sphinxstyleemphasis{Modern calculus and analytic geometry}, The 2002 Dover corrected republication of the original 1969 Macmillan Company edition, Dover Publications, USA: Chapters 7 and 14.

\item {} 
\sphinxAtStartPar
Spivak, M (2006), \sphinxstyleemphasis{Calculus (third edition)}, Cambridge University Press, The United Kingdom: Chapters 1, 2, 3, 4, 16, 18, and 19.

\item {} 
\sphinxAtStartPar
Thomas, GB Jr, and RL Finney (1996), \sphinxstyleemphasis{Calculus and analytic geometry (ninth edition)}, The 1998 corrected reprint version, Addison\sphinxhyphen{}Wesley Publishing Company, USA: Chapters P, 6, and 12.

\end{itemize}

\sphinxAtStartPar
More advanced references:
\begin{itemize}
\item {} 
\sphinxAtStartPar
Banks, J, G Elton and J Strantzen (2009), \sphinxstyleemphasis{Topology and analysis: Unit text for MAT3TA (2009 and 2010 edition)}, Department of Mathematics and Statistics, La Trobe University, Bundoora, February.

\item {} 
\sphinxAtStartPar
Corbae, D, MB Stinchcombe and J Zeman (2009), \sphinxstyleemphasis{An introduction to mathematical analysis for economic theory and econometrics}, Princeton University Press, USA: Chapter 2 (pp. 15\sphinxhyphen{}71).

\item {} 
\sphinxAtStartPar
Kolmogorov, AN and SV Fomin (1970), \sphinxstyleemphasis{Introductory real analysis}, Translated and Edited by RA Silverman, The 1975 Dover Edition (an unabridged, slightly corrected republication of the original 1970 Prentice\sphinxhyphen{}Hall edition), Dover Publications, USA: Chapter 1 (pp. 1\sphinxhyphen{}36).

\item {} 
\sphinxAtStartPar
Simon, C, and L Blume (1994), \sphinxstyleemphasis{Mathematics for economists}, WW Norton and Co, USA: Chapters 2 and 13 (pp. 10\sphinxhyphen{}38 and 273\sphinxhyphen{}299).

\end{itemize}

\sphinxAtStartPar
Websites:

\sphinxAtStartPar
The following websites contain discussions of the concept of relatively prime, or co\sphinxhyphen{}prime, numbers and polynomials. This is relevant to the topic of rational functions and partial fractions.
\begin{itemize}
\item {} 
\sphinxAtStartPar
\sphinxurl{https://www.mathsisfun.com/definitions/relatively-prime.html}

\item {} 
\sphinxAtStartPar
\sphinxurl{http://mathworld.wolfram.com/RelativelyPrime.html}

\item {} 
\sphinxAtStartPar
\sphinxurl{https://www.varsitytutors.com/hotmath/hotmath\_help/topics/relatively-prime}

\end{itemize}


\section{Mappings}
\label{\detokenize{03.mappings_functions_correspondences:mappings}}
\sphinxAtStartPar
Let \(X\) and \(Y\) be two sets. A rule \(f\) that assigns one or more elements of \(Y\) to each element of \(X\) is called a \sphinxstylestrong{mapping} from \(X\) into \(Y\). It is denoted by \(f: X \rightarrow Y\).

\sphinxAtStartPar
The set \(X\) is known as the \sphinxstylestrong{domain} of the mapping \(f\). The mapping must be defined for every element of \(X\). This means that \(x \in X \implies f(x) \in Y\).

\sphinxAtStartPar
The set \(Y\) is known as the \sphinxstylestrong{co\sphinxhyphen{}domain} of the mapping \(f\). Mappings are not required to generate \(Y\) from \(X\). This means that there might exist one or more elements \(y \in Y\) such that \(y \ne f(x)\) for any \(x \in X\).

\sphinxAtStartPar
The set of values \(y \in Y\) that can be generated from \(X\) by the function \(f\) is known as the \sphinxstylestrong{image} of \(X\) under \(f\). Sometimes the image of \(X\) under \(f\) is referred to as the \sphinxstylestrong{range} of \(f\). We will denote the range of \(f\) by \(f(X)\).

\begin{figure}[htbp]
\centering
\capstart

\noindent\sphinxincludegraphics[width=0.800\linewidth]{{mapping_diagram}.png}
\caption{Diagrammatic representation of a mapping}\label{\detokenize{03.mappings_functions_correspondences:id2}}\end{figure}


\subsection{Images}
\label{\detokenize{03.mappings_functions_correspondences:images}}
\sphinxAtStartPar
Consider a mapping \(f: X \rightarrow Y\).
The \sphinxstyleemphasis{image of the point} \(x \in X\) under the mapping \(f\) is the point, or collection of points, given by \(f(x) \in Y\).

\sphinxAtStartPar
The \sphinxstyleemphasis{image of the set} \(A \subseteq X\) under the mapping \(f\) is the set \(f(A) = \{f(x) \in Y : x \in A\}\).
\begin{itemize}
\item {} 
\sphinxAtStartPar
Clearly \(f(A) \subseteq Y\).

\end{itemize}

\sphinxAtStartPar
The \sphinxstyleemphasis{image of the domain} (\(X\)) under \(f\) is the set \(f(X) = \{f(x) \in Y : x \in X \}\).
\begin{itemize}
\item {} 
\sphinxAtStartPar
Clearly \(f(X) \subseteq Y\).

\end{itemize}

\sphinxAtStartPar
Note that if \(f: X \rightarrow Y\) and \(A \subseteq X\), then \(f(A) \subseteq f(X) \subseteq Y\).


\subsection{Pre\sphinxhyphen{}images}
\label{\detokenize{03.mappings_functions_correspondences:pre-images}}
\sphinxAtStartPar
Consider a mapping \(f: X \rightarrow Y\).

\sphinxAtStartPar
The \sphinxstylestrong{pre\sphinxhyphen{}image} \sphinxstyleemphasis{of the point} \(y \in Y\) under the mapping \(f\) is the point, or collection of points, in \(X\) for which \(y = f(x)\).

\sphinxAtStartPar
It is possible for a point \(y \in Y\) to have either
\begin{itemize}
\item {} 
\sphinxAtStartPar
no pre\sphinxhyphen{}image

\item {} 
\sphinxAtStartPar
a unique pre\sphinxhyphen{}image, or

\item {} 
\sphinxAtStartPar
multiple pre\sphinxhyphen{}images.
If a point \(y \in Y\) has a unique pre\sphinxhyphen{}image under the mapping \(f\), then it is denoted by \(f^{−1}(y)\).

\end{itemize}

\sphinxAtStartPar
The \sphinxstyleemphasis{pre\sphinxhyphen{}image of the set} \(B \subseteq Y\) under \(f\) is the set \(f^{−1}(B) = \{x \in X : f(x) \in B\}\).

\sphinxAtStartPar
Recall that it is possible for there to exist \(y \in Y\) such that \(y \ne f(x)\) for any \(x \in X\). If the set B consists entirely of such points, then \(f^{−1}(B) = \varnothing\).

\sphinxAtStartPar
Since \(f(x) \in Y\) for all \(x \in X\), it must be the case that \(f^{-1}(Y) = X\).


\subsection{Types of mappings}
\label{\detokenize{03.mappings_functions_correspondences:types-of-mappings}}
\sphinxAtStartPar
There are four basic types of mappings. These are as follows.
\begin{itemize}
\item {} 
\sphinxAtStartPar
A one\sphinxhyphen{}to\sphinxhyphen{}one mapping:
\begin{itemize}
\item {} 
\sphinxAtStartPar
Each point in \(X\) has a unique image in \(Y\); and

\item {} 
\sphinxAtStartPar
Each point in \(Y\) has either a unique pre\sphinxhyphen{}image in \(X\) or no pre\sphinxhyphen{}image in \(X\).

\end{itemize}

\item {} 
\sphinxAtStartPar
A many\sphinxhyphen{}to\sphinxhyphen{}one mapping:
\begin{itemize}
\item {} 
\sphinxAtStartPar
Each point in \(X\) has a unique image in \(Y\); but

\item {} 
\sphinxAtStartPar
At least one point in \(Y\) has multiple pre\sphinxhyphen{}images in \(X\).

\end{itemize}

\item {} 
\sphinxAtStartPar
A one\sphinxhyphen{}to\sphinxhyphen{}many mapping:
\begin{itemize}
\item {} 
\sphinxAtStartPar
At least one point in \(X\) has multiple images in \(Y\); but

\item {} 
\sphinxAtStartPar
Each point in \(Y\) has either a unique pre\sphinxhyphen{}image in \(X\) or no pre\sphinxhyphen{}image in \(X\).

\end{itemize}

\item {} 
\sphinxAtStartPar
A many\sphinxhyphen{}to\sphinxhyphen{}many mapping:
\begin{itemize}
\item {} 
\sphinxAtStartPar
At least one point in \(X\) has multiple images in \(Y\); and

\item {} 
\sphinxAtStartPar
At least one point in \(Y\) has multiple pre\sphinxhyphen{}images in \(X\).

\end{itemize}

\end{itemize}

\sphinxAtStartPar
Mappings whose domain points all have unique images are known as
\sphinxstylestrong{functions}. In other words, one\sphinxhyphen{}to\sphinxhyphen{}one mappings and many\sphinxhyphen{}to\sphinxhyphen{}one mappings are
known as functions.

\sphinxAtStartPar
A one\sphinxhyphen{}to\sphinxhyphen{}one function \(f: X \rightarrow Y\) is called an \sphinxstylestrong{injection}. The pre\sphinxhyphen{}image of a one\sphinxhyphen{}to\sphinxhyphen{}one function is known as its \sphinxstylestrong{inverse}.

\sphinxAtStartPar
Mappings that have at least one domain point with multiple images are known as \sphinxstylestrong{correspondences}. In other words, one\sphinxhyphen{}to\sphinxhyphen{}many mappings and many\sphinxhyphen{}to\sphinxhyphen{}many mappings are known as correspondences.

\sphinxAtStartPar
Note that we could express a correspondence of the form \(f : X \rightarrow Y\) as a function of the form \(g: X \rightarrow 2^Y\).


\subsection{Into and onto}
\label{\detokenize{03.mappings_functions_correspondences:into-and-onto}}
\sphinxAtStartPar
Consider a mapping \(f : X \rightarrow Y\). Clearly we must have \(f(X) \subseteq Y\).
\begin{itemize}
\item {} 
\sphinxAtStartPar
If \(f(X ) \subset Y\), so that \(f(X) \ne Y\), we say that \(f\) maps \(X\) “into” \(Y\).

\item {} 
\sphinxAtStartPar
If \(f(X) = Y\), we say that \(f\) maps \(X\) “onto” \(Y\).
\begin{itemize}
\item {} 
\sphinxAtStartPar
If \(f(X) = Y\) and \(f\) is a function, then we call \(f\) a \sphinxstylestrong{surjection}.

\end{itemize}

\end{itemize}

\sphinxAtStartPar
A mapping that is both an injection and a surjection is called a \sphinxstylestrong{bijection}. In other words, a function that is \sphinxstyleemphasis{both one\sphinxhyphen{}to\sphinxhyphen{}one and onto} is called a
bijection.


\subsection{Examples}
\label{\detokenize{03.mappings_functions_correspondences:examples}}
\sphinxAtStartPar
Consider the mapping \(f: \mathbb{R} \rightarrow \mathbb{R}\) defined by \(f(x) = x\). This is sometimes called the identity map. Note that \(f\) is both onto and one\sphinxhyphen{}to\sphinxhyphen{}one. This means that \(f\) is a bijection. It also means that the pre\sphinxhyphen{}image of \(f\) is an inverse function.

\sphinxAtStartPar
Consider the mapping \(f: \mathbb{R} \rightarrow \mathbb{R}\) defined by \(f(x) = x^2\). Note that \(f\) is into, but not onto, because \(f^{−1}(R_{−−}) = \varnothing\), where \(\mathbb{R}_{−−} = \{x \in \mathbb{R} : x < 0\}\).

\sphinxAtStartPar
Note also that f is many\sphinxhyphen{}to\sphinxhyphen{}one, and hence not one\sphinxhyphen{}to\sphinxhyphen{}one, because \(2 \in \mathbb{R}, (−2) \in \mathbb{R}, 2 \ne (−2)\), and both \(2^2 = 4\) and \((−2)^2 = 4\).
This means that f is neither an injection nor a surjection. Thus it cannot be a bijection. It also means that the pre\sphinxhyphen{}image of f is not an inverse function.


\subsection{Composite functions}
\label{\detokenize{03.mappings_functions_correspondences:composite-functions}}
\sphinxAtStartPar
Let \(f : X \rightarrow Y\) be a function of the form \(y = f(x)\).
Let \(g : Y \rightarrow Z\) be a function of the form \(z = g(y)\).
The composite function \(h = g \circ f\) is defined by \(h(x) = g(f(x))\).
The composite function \(h = g \circ f\) is a mapping \(h : X \rightarrow Z\) of the
form \(z = h(x)\).

\sphinxAtStartPar
Example:
Let \(f : \mathbb{R}^2_{++} \rightarrow \mathbb{R}_{++}\) be defined by
\(f(x_1, x_2) = x_1^\alpha \; x_2^{(1 − \alpha)}\)
for some fixed \(\alpha \in (0, 1)\).

\sphinxAtStartPar
Let \(g : \mathbb{R}_{++} \rightarrow \mathbb{R}\) be defined by \(g(x) = ln(x)\).

\sphinxAtStartPar
Then the composite function \(h = g \circ f\) is a mapping \(h: \mathbb{R}^2_{++} \rightarrow \mathbb{R}\) that is defined by \(h(x_1, x_2) = g(f(x_1, x_2)) = ln(x_1^\alpha \; x_2^{(1 − \alpha)}) = \alpha \; ln(x_1) + (1 − \alpha) \; ln(x_2)\)


\subsection{One\sphinxhyphen{}to\sphinxhyphen{}one functions}
\label{\detokenize{03.mappings_functions_correspondences:one-to-one-functions}}
\sphinxAtStartPar
A function \(f : X \rightarrow Y\) is \sphinxstylestrong{one\sphinxhyphen{}to\sphinxhyphen{}one} if
\begin{equation*}
\begin{split}x \ne y \iff f(x) \ne f(y)\end{split}
\end{equation*}
\sphinxAtStartPar
The contra\sphinxhyphen{}positive version of this condition is that
\begin{equation*}
\begin{split}f(x) = f(y) \iff x = y\end{split}
\end{equation*}
\sphinxAtStartPar
Examples:
\begin{itemize}
\item {} 
\sphinxAtStartPar
\(f : \mathbb{R} \rightarrow \mathbb{R}\) defined by \(f(x ) = x\) is a one\sphinxhyphen{}to\sphinxhyphen{}one function.

\item {} 
\sphinxAtStartPar
\(f : \mathbb{R} \rightarrow \mathbb{R}\) defined by \(f(x ) = x^2\) is not a one\sphinxhyphen{}to\sphinxhyphen{}one function.

\item {} 
\sphinxAtStartPar
\(f : \mathbb{R}_+ \rightarrow \mathbb{R}\) defined by \(f(x) = x^2\) is a one\sphinxhyphen{}to\sphinxhyphen{}one function.

\end{itemize}


\subsection{Non\sphinxhyphen{}decreasing and strictly increasing functions}
\label{\detokenize{03.mappings_functions_correspondences:non-decreasing-and-strictly-increasing-functions}}
\sphinxAtStartPar
A function \(f : X \rightarrow \mathbb{R}\), where \(X \subseteq \mathbb{R}\), is said to be a \sphinxstylestrong{non\sphinxhyphen{}decreasing} function if
\begin{equation*}
\begin{split}x \leqslant y \iff f(x) \leqslant f(y)\end{split}
\end{equation*}
\sphinxAtStartPar
A function \(f : X \rightarrow \mathbb{R}\), where \(X \subseteq \mathbb{R}\), is said to be a \sphinxstylestrong{strictly increasing} function if both
\begin{itemize}
\item {} 
\sphinxAtStartPar
(a) \(x = y \iff f(x) = f(y)\); and

\item {} 
\sphinxAtStartPar
(b) \(x < y \iff f(x) < f(y)\).

\end{itemize}

\sphinxAtStartPar
Note the following:
\begin{itemize}
\item {} 
\sphinxAtStartPar
A strictly increasing function is also a one\sphinxhyphen{}to\sphinxhyphen{}one function.

\item {} 
\sphinxAtStartPar
There are some one\sphinxhyphen{}to\sphinxhyphen{}one functions that are not strictly increasing.

\item {} 
\sphinxAtStartPar
A strictly increasing function is also a non\sphinxhyphen{}decreasing function.

\item {} 
\sphinxAtStartPar
There are some non\sphinxhyphen{}decreasing functions that are not strictly increasing.

\end{itemize}


\subsection{Economic application: utility functions are not unique}
\label{\detokenize{03.mappings_functions_correspondences:economic-application-utility-functions-are-not-unique}}
\sphinxAtStartPar
Suppose that \(U : X \rightarrow \mathbb{R}\) is a utility function that represents the weak preference relation \(\succsim\). This means that \(x \succsim y \iff U(x) \geqslant U(y)\).

\sphinxAtStartPar
Let \(f: \mathbb{R} \rightarrow \mathbb{R}\) be a strictly increasing function. Consider the composite function \(V = f \circ U\). It can be shown that \(V\) is also a utility function that represents the weak preference relation \(\succsim\). In other words, it can be shown that \(x \succsim y \iff V(x) \geqslant V(y)\).


\subsection{Example: A Cobb\sphinxhyphen{}Douglas utility function}
\label{\detokenize{03.mappings_functions_correspondences:example-a-cobb-douglas-utility-function}}
\sphinxAtStartPar
Consider a consumer whose preferences over bundles of strictly positive amounts of each of two commodities can be represented by a utility function \(U : \mathbb{R}^2_{++} \rightarrow \mathbb{R}_{++}\) of the form
\begin{equation*}
\begin{split}U(x_1, x_2) = Ax_1^\alpha x_2^\beta\end{split}
\end{equation*}
\sphinxAtStartPar
where \(A > 0, \alpha > 0\), and \(\beta > 0\).

\sphinxAtStartPar
Such preferences are known as Cobb\sphinxhyphen{}Douglas preferences.
\begin{itemize}
\item {} 
\sphinxAtStartPar
The function \(f: \mathbb{R}_{++} \rightarrow \mathbb{R}_{++}\) defined by \(f(x) = \left( \frac{1}{A} \right)x\) is strictly increasing. Thus we know that another utility function that represents this consumer’s preferences is

\end{itemize}
\begin{equation*}
\begin{split}V(x_1, x_2) = f(U((x_1, x_2)) = \left( \frac{1}{A} \right) (Ax_1^\alpha x_2^\beta) 
= x_1^\alpha x_2^\beta\end{split}
\end{equation*}\begin{itemize}
\item {} 
\sphinxAtStartPar
The function \(g: \mathbb{R}_{++} \rightarrow \mathbb{R}_{++}\) defined by \(g(x) = x^{\frac{1}{(\alpha + \beta)}}\) is strictly increasing. (If any relevant surd expression can be either positive or negative, then we will assume that the positive option is chosen throughout this Cobb\sphinxhyphen{}Douglas preferences example.) Thus we know that another utility function that represents this consumer’s preferences is

\end{itemize}
\begin{equation*}
\begin{split}W(x_1, x_2) = g(V((x_1, x_2)) = (x_1^\alpha x_2^\beta)^{\frac{1}{(\alpha + \beta)}}
= x_1^\gamma x_2^{(1 - \gamma)}\end{split}
\end{equation*}
\sphinxAtStartPar
where \( \gamma = \frac{\alpha}{\alpha + \beta} \in (0, 1)\).
\begin{itemize}
\item {} 
\sphinxAtStartPar
The function \(k: \mathbb{R}_{++} \rightarrow \mathbb{R}\) defined by \(k(x) = ln(x )\) is strictly increasing. Thus we know that another utility function that represents this consumer’s preferences is

\end{itemize}
\begin{equation*}
\begin{split}Z(x_1, x_2) = k(W((x_1, x_2)) = ln (x_1^\gamma x_2^{(1 - \gamma)})\end{split}
\end{equation*}\begin{equation*}
\begin{split}= \gamma \; ln(x_1) + (1 − \gamma) \; ln(x_2) \end{split}
\end{equation*}

\section{Some types of functions}
\label{\detokenize{03.mappings_functions_correspondences:some-types-of-functions}}
\sphinxAtStartPar
Some types of univariate functions include the following:
\begin{itemize}
\item {} 
\sphinxAtStartPar
Polynomial functions
\begin{itemize}
\item {} 
\sphinxAtStartPar
These include constant functions, linear (and affine) functions, quadratic functions, and some power functions as special cases.

\end{itemize}

\item {} 
\sphinxAtStartPar
Exponentional functions

\item {} 
\sphinxAtStartPar
Power functions

\item {} 
\sphinxAtStartPar
Logarithmic functions

\item {} 
\sphinxAtStartPar
Trigonometric functions
\begin{itemize}
\item {} 
\sphinxAtStartPar
We are unlikely to have enough time to cover trigonometric functions in this course.
There are also multivariate versions of these types of functions.

\end{itemize}

\end{itemize}


\subsection{Polynomial functions}
\label{\detokenize{03.mappings_functions_correspondences:polynomial-functions}}
\sphinxAtStartPar
A \sphinxstylestrong{polynomial function} (of one variable) is a function of the form
\begin{equation*}
\begin{split}
\begin{align*}
f(x) &= \sum_{i = 0}^n a_i x^i
\\
&= a_n x^n + a_{n − 1} x^{n − 1} + · · · + a_1 x^1 + a_0 x^0
\\
&= a_n x^n + a_{n − 1} x^{n − 1} + · · · + a_1 x + a_0
\end{align*}
\end{split}
\end{equation*}
\sphinxAtStartPar
In order to distinguish between different types of polynomials, we will typically assume that the coefficient on the term with the highest power of the variable \(x\) is non\sphinxhyphen{}zero. The only exception is the case in which this term involves \(x^0 = 1\), in which case we will allow both \(a_0 \ne 0\) and \(a_0 = 0\).


\subsubsection{Examples}
\label{\detokenize{03.mappings_functions_correspondences:id1}}\begin{itemize}
\item {} 
\sphinxAtStartPar
A constant (degree zero) polynomial (\(a_0 \ne 0\) or \(a_0 = 0\)):

\end{itemize}
\begin{equation*}
\begin{split}
f(x) = a_0
\end{split}
\end{equation*}\begin{itemize}
\item {} 
\sphinxAtStartPar
A linear (degree one) polynomial (\(a_1 \ne 0\)):

\end{itemize}
\begin{equation*}
\begin{split}
f(x) = a_1 x + a_0
\end{split}
\end{equation*}
\sphinxAtStartPar
It is sometimes useful to distinguish between linear functions and affine functions. See below for details.
\begin{itemize}
\item {} 
\sphinxAtStartPar
A quadratic (degree two) polynomial (\(a_2 \ne 0\)):

\end{itemize}
\begin{equation*}
\begin{split}
f(x) = a_2 x^2 + a_1 x + a_0
\end{split}
\end{equation*}\begin{itemize}
\item {} 
\sphinxAtStartPar
A cubic (degree three) polynomial (\(a_3 \ne 0\)):

\end{itemize}
\begin{equation*}
\begin{split}
f(x) = a_3 x^3 + a_2 x^2 + a_1 x + a_0
\end{split}
\end{equation*}

\subsection{Affine functions and linear functions}
\label{\detokenize{03.mappings_functions_correspondences:affine-functions-and-linear-functions}}
\sphinxAtStartPar
We often loosely speak about a linear function of one variable being a function of the form \(f(x) = a_1 x + a_0\), where \(a_0 \in \mathbb{R}\) and \(a_1 \in \mathbb{R} \setminus \{0\}\) are fixed parameters, and \(x \in \mathbb{R}\) is the single variable.

\sphinxAtStartPar
Sometimes, we want to be more precise than this in order to identify the special case in which \(a_0 = 0\). In such cases, we call a function of the general form \(f(x) = a_1 x + a_0\), in which \(a_0 \in \mathbb{R}\), an \sphinxstylestrong{affine function}; and a function of the specific form \(f(x) = a_1 x\), in which \(a_0 = 0\), a \sphinxstylestrong{linear function}.

\sphinxAtStartPar
Using this more precise terminology, the family of linear functions is a proper subset of the family of affine functions. Note that we assume that \(a_1 \in \mathbb{R} \setminus \{0\}\) in both cases.


\subsection{Exponential functions}
\label{\detokenize{03.mappings_functions_correspondences:exponential-functions}}
\sphinxAtStartPar
An \sphinxstylestrong{exponential function} is a non\sphinxhyphen{}linear function in which the independent variable appears as an exponent.

\sphinxAtStartPar
An example of an exponential function is
\begin{equation*}
\begin{split}
f(x) = Ca^x
\end{split}
\end{equation*}
\sphinxAtStartPar
where \(C\) is a fixed parameter (called the \sphinxstylestrong{coefficient}), \(a \in \mathbb{R}\) is a fixed parameter (called the \sphinxstylestrong{base}), and \(x \in \mathbb{R}\) is an independent variable (called the \sphinxstylestrong{exponent}).
\begin{itemize}
\item {} 
\sphinxAtStartPar
Note that if \(a = 0\), then \(f(x)\) is only defined for \(x > 0\).

\item {} 
\sphinxAtStartPar
Note also that if \(a < 0\), then sometimes \(f(x) \notin \mathbb{R}\). An example of this is the case when \(C = 1, a = (−1)\) and \(x = \frac{1}{2}\). In this case, we have:

\end{itemize}
\begin{equation*}
\begin{split}
f(\frac{1}{2}) = (1)(−1)^{\frac{1}{2}} = \sqrt{−1} = i \notin \mathbb{R}
\end{split}
\end{equation*}

\subsubsection{Popular choices of base}
\label{\detokenize{03.mappings_functions_correspondences:popular-choices-of-base}}
\sphinxAtStartPar
Two popular choices of base for exponential functions are \(a = 10\) and \(a = e\), where \(e\) denotes Euler’s constant. Euler’s constant is defined to be the number
\begin{equation*}
\begin{split}
e = \lim_{n \rightarrow \infty} \left(1 + \frac{1}{n} \right)^n
\end{split}
\end{equation*}
\sphinxAtStartPar
Note that Euler’s constant is an irrational number. This means that it is a real number that cannot be represented as the ratio of an integer to a natural number.

\sphinxAtStartPar
The function \(f(x) = e^x\) is sometimes called “the” exponential function.


\subsubsection{Exponential arithmetic}
\label{\detokenize{03.mappings_functions_correspondences:exponential-arithmetic}}
\sphinxAtStartPar
Assuming that the expressions are well\sphinxhyphen{}defined, we have the following laws of exponential arithmetic.
\begin{itemize}
\item {} 
\sphinxAtStartPar
The power of zero: \(a_0 = 1\) if \(a \ne 0\), while \(a^0\) is undefined if \(a = 0\).

\item {} 
\sphinxAtStartPar
Multiplication of two exponential functions with the same base:

\end{itemize}
\begin{equation*}
\begin{split}
(Ca^x)(Da^y) = CD a^{(x + y)}
\end{split}
\end{equation*}\begin{itemize}
\item {} 
\sphinxAtStartPar
Division of two exponential functions with the same base:

\end{itemize}
\begin{equation*}
\begin{split}
\frac{(Ca^x)}{(Da^y)} = \frac{C}{D} a^{(x - y)}
\end{split}
\end{equation*}\begin{itemize}
\item {} 
\sphinxAtStartPar
An exponential function whose base is an exponential function:

\end{itemize}
\begin{equation*}
\begin{split}
(Ca^x)^y = C^y a^{xy}
\end{split}
\end{equation*}

\subsection{Power functions}
\label{\detokenize{03.mappings_functions_correspondences:power-functions}}
\sphinxAtStartPar
A \sphinxstylestrong{power function} takes the form
\begin{equation*}
\begin{split}
f(x) = Cx^a
\end{split}
\end{equation*}
\sphinxAtStartPar
where \(C \in \mathbb{R}\) is a fixed parameter, \(a \in \mathbb{R}\) is a fixed parameter, and \(x \in \mathbb{R}\) is an independent variable.
\begin{itemize}
\item {} 
\sphinxAtStartPar
Note that when \(a \in \mathbb{N}\), a power function can also be viewed as a polynomial function with a single term.

\item {} 
\sphinxAtStartPar
Note that a power function can also be viewed as a type of exponential expression in which the base is \(x\) and the exponent is \(a\). This means that the laws of exponential arithmetic carry over to “power function arithmetic”.

\end{itemize}


\subsubsection{Power function arithmetic}
\label{\detokenize{03.mappings_functions_correspondences:power-function-arithmetic}}
\sphinxAtStartPar
Assuming that the expressions are well\sphinxhyphen{}defined, we have the following laws of power function arithmetic.
\begin{itemize}
\item {} 
\sphinxAtStartPar
The power of zero: \(x_0 = 1\) if \(x \ne 0\), while \(x^0\) is undefined if \(x = 0\).

\item {} 
\sphinxAtStartPar
Multiplication of two power functions with the same base:

\end{itemize}
\begin{equation*}
\begin{split}
(Cx^a)(Dx^b) = CD x^{(a + b)}
\end{split}
\end{equation*}\begin{itemize}
\item {} 
\sphinxAtStartPar
Division of two power functions with the same base:

\end{itemize}
\begin{equation*}
\begin{split}
\frac{(Cx^a)}{(Dx^b)} = \frac{C}{D} x^{(a - b)}
\end{split}
\end{equation*}\begin{itemize}
\item {} 
\sphinxAtStartPar
A power function whose base is itself a power function:

\end{itemize}
\begin{equation*}
\begin{split}
(Cx^a)^b = C^b x^{ab}
\end{split}
\end{equation*}

\subsubsection{A rectangular hyperbola}
\label{\detokenize{03.mappings_functions_correspondences:a-rectangular-hyperbola}}
\sphinxAtStartPar
Consider the function \(f : \mathbb{R} \setminus \{0\} \rightarrow \mathbb{R}\) defined by \(f(x) = \frac{a}{x}\), where \(a \ne 0\). This is a special type of power function, as can be seen by noting that it can be rewritten as \(f(x) = ax^{−1}\). The equation for the graph of this function is
\begin{equation*}
\begin{split}
y = \frac{a}{x}
\end{split}
\end{equation*}
\sphinxAtStartPar
Note that this equation can be rewritten as \(xy = a\). This is the equation of a rectangular hyperbola.

\sphinxAtStartPar
\sphinxstyleemphasis{Graph it on the whiteboard for both the case where \(a > 0\) and the case where \(a < 0\).}
\begin{itemize}
\item {} 
\sphinxAtStartPar
Economic application: A constant “own\sphinxhyphen{}price elasticity of demand” demand curve.

\end{itemize}


\subsection{Logarithms}
\label{\detokenize{03.mappings_functions_correspondences:logarithms}}
\sphinxAtStartPar
A \sphinxstylestrong{logarithm} undoes an exponent.
Thus we have
\begin{equation*}
\begin{split}
log_a (a^x) = x
\end{split}
\end{equation*}
\sphinxAtStartPar
The expression \(log_a\) stands for “log base \(a\)” or “logarithm base \(a\)”. Popular choices of base are \(a = 10\) and \(a = e\).

\sphinxAtStartPar
The function
\begin{equation*}
\begin{split}
g(x) = log_a(x)
\end{split}
\end{equation*}
\sphinxAtStartPar
is the “logarithm base \(a\)” function. The “logarithm base \(a\)” function is the inverse for the “exponential base \(a\)” function. The reason for this is that
\begin{equation*}
\begin{split}
g(f(x)) = g(a^x) = log_a(a^x) = x
\end{split}
\end{equation*}

\subsubsection{Natural (or Naperian) logarithms}
\label{\detokenize{03.mappings_functions_correspondences:natural-or-naperian-logarithms}}
\sphinxAtStartPar
A “logarithm base \sphinxstyleemphasis{e}” is known as a natural, or Naperian, logarithm. It is named after John Napier. See Shannon (1995, pp. 270\sphinxhyphen{}271) for a brief introduction to John Napier. The standard notation for a \sphinxstylestrong{natural logarithm} is \(ln\), although you could also use \(log_e\).

\sphinxAtStartPar
The function
\begin{equation*}
\begin{split} g(x) = ln(x) \end{split}
\end{equation*}
\sphinxAtStartPar
is the “logarithm base \(e\)” function. The natural logarithm function is the inverse function for “the” exponential function, since
\begin{equation*}
\begin{split} g(f(x)) = g(e^x) = ln(e^x) = log_e(e^x) = x \end{split}
\end{equation*}
\sphinxAtStartPar
\sphinxstyleemphasis{Illustrate the inverse (or “reflection through the 45 degree (y = x) line”) relationship between ln (x ) and ex on the whiteboard.}


\subsubsection{Logarithmic arithmetic}
\label{\detokenize{03.mappings_functions_correspondences:logarithmic-arithmetic}}
\sphinxAtStartPar
Assuming that the expressions are well\sphinxhyphen{}defined, we have the following
laws of logarithmic arithmetic.
\begin{itemize}
\item {} 
\sphinxAtStartPar
Multiplication of two logarithmic functions with the same base:

\end{itemize}
\begin{equation*}
\begin{split} log_a(xy) = log_a(x) + log_a(y) \end{split}
\end{equation*}\begin{itemize}
\item {} 
\sphinxAtStartPar
Division of two logarithmic functions with the same base:

\end{itemize}
\begin{equation*}
\begin{split}log_a \left( \frac{x}{y} \right) =  log_a(x) − log_a(y) \end{split}
\end{equation*}\begin{itemize}
\item {} 
\sphinxAtStartPar
A logarithmic function whose argument is an exponential function:

\end{itemize}
\begin{equation*}
\begin{split}log_a(x^y) = y \; log_a(x)\end{split}
\end{equation*}
\sphinxAtStartPar
Note that
\begin{equation*}
\begin{split}log_a(a) = log_a(a^1) = 1\end{split}
\end{equation*}

\subsection{Rational functions}
\label{\detokenize{03.mappings_functions_correspondences:rational-functions}}
\sphinxAtStartPar
A \sphinxstylestrong{rational function} \(R(x)\) is simply the ratio of two polynomial functions, \(P(x)\) and \(Q(x)\). It takes the form
\begin{equation*}
\begin{split}
R(x) = \frac{P(x)}{Q(x)} = 
\frac{a_mx^m + a_{m−1} x^{m−1} + · · · + a_1 x + a_0}
{b_n x^n + b_{n−1} x^{n−1} + · · · + b_1 x + b_0}
\end{split}
\end{equation*}
\sphinxAtStartPar
where
\begin{equation*}
\begin{split} P(x) = a_mx^m + a_{m−1} x^{m−1} + · · · + a_1 x + a_0 \end{split}
\end{equation*}
\sphinxAtStartPar
is an \(m\)\sphinxhyphen{}th order polynomial (so that \(a_m \ne 0\)), and
\begin{equation*}
\begin{split} Q(x) =  b_n x^n + b_{n−1} x^{n−1} + · · · + b_1 x + b_0 \end{split}
\end{equation*}
\sphinxAtStartPar
is an \(n\)\sphinxhyphen{}th order polynomial (so that \(b_n \ne 0\)).
\begin{itemize}
\item {} 
\sphinxAtStartPar
Note that there is no requirement that the polynomial functions \(P(x)\) and \(Q(x)\) be of the same order. (In other words, we do not require that \(m = n\).)

\item {} 
\sphinxAtStartPar
The most interesting case is when \(m < n\). In such cases, the rational function \(R(x)\) is called a \sphinxstylestrong{“proper” rational function}.

\item {} 
\sphinxAtStartPar
When \(m > n\), then we can always use the process of long division to write the original rational function \(R(x)\) as the sum of a polynomial function \(Y(x)\) and another proper rational function \(R^∗(x)\).

\end{itemize}

\sphinxAtStartPar
This is nicely illustrated by the following example from Chapter 14 of Silverman (1969).

\sphinxAtStartPar
Consider the rational function \(R(x) = \frac{x^2 + x − 1}{x − 1}\). Note the following.

\sphinxAtStartPar
Thus we have
\begin{equation*}
\begin{split}
R(x) = \frac{x^2 + x − 1}{x − 1} = (x + 2) + \left( \frac{1}{x − 1} \right)
\end{split}
\end{equation*}
\sphinxAtStartPar
If \(R(x)\) is a non\sphinxhyphen{}proper rational function, then it can be written as the sum of a polynomial function (\(Y(x)\)) and a proper rational function (\(R^∗(x)\)), so that \(R(x) = Y(x) + R^∗(x)\).  In the example above, we had \(R(x) = \frac{x^2 + x − 1}{x − 1}\), \(Y(x) = (x + 2)\), and \(R^∗(x) = \left( \frac{1}{x − 1} \right)\), so that
\begin{equation*}
\begin{split}
\frac{x^2 + x − 1}{x − 1} = (x + 2) + \left( \frac{1}{x − 1} \right)
\end{split}
\end{equation*}
\sphinxAtStartPar
The proper rational function component is sometimes known as a “remainder” term.

\sphinxAtStartPar
In cases where \(R(x)\) is a proper rational function to begin with, then we have \(Y(x) = 0\), leaving us only with the remainder term.

\sphinxAtStartPar
The remainder term \(R^∗(x)\) can sometimes be converted into a more convenient form by using the technique of “Partial Fractions”.


\subsection{Partial fractions}
\label{\detokenize{03.mappings_functions_correspondences:partial-fractions}}
\sphinxAtStartPar
Consider two rational functions, \(R_1(x) = \frac{P_1(x)}{Q_1(x)}\) and \(R_2(x) = \frac{P_2(x)}{Q_2(x)}\).  Note that the sum of these two rational functions forms a third rational function, since
\begin{equation*}
\begin{split}
R_3(x) = R_1(x) + R_2(x) 
= \frac{P_1(x)}{Q_1(x)} + \frac{P_2(x)}{Q_2(x)} \\
= \frac{(P_1(x) Q_2(x) + P_2(x) Q_1(x))}{Q_1(x) Q_2(x)}
\end{split}
\end{equation*}
\sphinxAtStartPar
If we reverse this process, we can write the rational function \(\frac{(P_1(x) Q_2(x) + P_2(x) Q_1(x))}{Q_1(x) Q_2(x)}\) as the sum of two other rational functions, \(\frac{P_1(x)}{Q_1(x)}\) and \(\frac{P_2(x)}{Q_2(x)}\). This decomposition is known as a \sphinxstylestrong{“partial fractions” decomposition}.

\sphinxAtStartPar
Suppose that \(P(x)\) and \(Q(x)\) are \sphinxstylestrong{relatively prime} polynomials.
\begin{itemize}
\item {} 
\sphinxAtStartPar
Two polynomials (or, indeed, two integers) are said to be “relatively prime” (or “co\sphinxhyphen{}prime”) if their “highest common factor” (which is also known as their “greatest common divisor”) is the number “one”.
Suppose also that the degree of the polynomial M(x ) is strictly less
than the degree of the polynomial P(x )Q(x ).
This means that the rational function R∗(x ) = M(x )
P(x )Q(x ) is a proper
rational function.
Under these circumstances there exist two unique polynomial
functions, A(x ) and B(x ), such that
R∗(x ) = M(x )
P(x )Q(x ) = A(x )
P(x ) + B(x )
Q(x ) ,
where the degree of A(x ) is strictly less than the degree of P(x ) and
the degree of B(x ) is strictly less than the degree of Q(x ).

\end{itemize}







\renewcommand{\indexname}{Index}
\printindex
\end{document}